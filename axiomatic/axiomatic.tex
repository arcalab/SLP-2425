\documentclass{beamer}
\usepackage{etex} % fixes new-dimension error
\usepackage{lmodern}
\usepackage[T1]{fontenc}
\usetheme{metropolis}
\metroset{block=fill}
%\usetheme{Boadilla}

\setbeamertemplate{footline}
{
  \leavevmode%
  \hbox{%
  \begin{beamercolorbox}[wd=.4\paperwidth,ht=2.25ex,dp=1ex,center]{author in head/foot}%
    \usebeamerfont{author in head/foot}\insertshortauthor
  \end{beamercolorbox}%
  \begin{beamercolorbox}[wd=.5\paperwidth,ht=2.25ex,dp=1ex,center]{title in head/foot}%
    \usebeamerfont{title in head/foot}\insertsection
  \end{beamercolorbox}%
  \begin{beamercolorbox}[wd=.1\paperwidth,ht=2.25ex,dp=1ex,right]{date in head/foot}%
    \insertframenumber{} / \inserttotalframenumber\hspace*{2ex} 
  \end{beamercolorbox}}%
  \vskip0pt%
}
\makeatother
\setbeamertemplate{navigation symbols}{}
%----------------------------------------------------------------------------
\usepackage{graphicx,amsmath}
\usepackage{stmaryrd} % cf. interleave
\usepackage{booktabs}
\usepackage{amscd}
\usepackage{multicol}
\usepackage[absolute,overlay]{textpos}
\usepackage{alltt}
\usepackage{proof}
\usepackage{listings}
\usepackage[all]{xy}
\input{macros}
%------ using pstricks (rnode etc) ------------------------------------------
\usepackage{pstricks,pst-node,pst-text,pst-3d}

% ------ using color ---------------------------------------------------------
\newrgbcolor{goldenrod}{.80392 .60784 .11373}
\newrgbcolor{darkgoldenrod}{.5451 .39608 .03137}
\newrgbcolor{brown}{.15 .15 .15}
\newrgbcolor{darkolivegreen}{.33333 .41961 .18431}
\def\gold#1{{\goldenrod #1}}
\def\dgold#1{{\alert{#1}}}
\def\dkb#1{{\blue #1}}
\def\tdkb#1{\textbf{\darkblue #1}}
\def\gre#1{{\darkolivegreen #1}}
\def\gry#1{{\gray #1}}
\def\rdb#1{{\red #1}}
%----------------------------------------------------------------------------

\AtBeginSection[]
{
    \begin{frame}
        \frametitle{Table of Contents}
        \tableofcontents[currentsection]
    \end{frame}
}

\author[Renato Neves]{Renato Neves}

% logos of institutions
\titlegraphic{
  \begin{textblock*}{5cm}(6.5cm,7.4cm)
     \includegraphics[scale=0.06]{images/uminho.png}\hspace*{.85cm}~%
  \end{textblock*}
  \begin{textblock*}{5cm}(9.2cm,7.45cm)
    \includegraphics[scale=0.50]{images/haslab.pdf}
  \end{textblock*}
}

% No date
\date{}

\begin{document}

\title{Axiomatic Semantics}

\frame[plain]{\titlepage}


\begin{frame}{Semantics for Every Season}

        \makebox[\linewidth][c]{%
        \begin{tabular}{ l l }
        Operational semantics & How a program operates
        \\
        Denotational semantics & What a program is 
        \\
        \alert{\underline{Axiomatic semantics}} 
                               & Which logical properties it satisfies
        \end{tabular}
        }

\end{frame}

\section{Motivation}

\begin{frame}{A Brief Warm-up} 

        Solve the following exercises via your favorite semantics  
        \smallskip
        \begin{itemize}
                \item Calculate the output of $\prog{ x := 1 \> {\blue ;} \> x
                        := 2}$
                %
                \\[15pt]
                %
                \item Show that the following program outputs a state with
                $\prog{x} \geq 2$
                \begin{center}
                        $\prog{{\blue if} \> \> x = 1 \> \> {\blue then } \> \>
                        x := 2 \> \> {\blue else} \> \> x := 3}$
                %
                \\[15pt]
                %
                \end{center}
                %
                \item Show that the following program is the factorial
                function
                \begin{center}
                        $\prog{{\blue while} \> x > 0 \> \{ y := x \times y
                        \> {\blue ;} \> x := x - 1 \} }$
                \end{center}
        \end{itemize} 

        \pause
        \bigskip
        \textbf{Hard ?}
\end{frame}

\begin{frame}{The Right Tools}

        Two last exercises were about \alert{\underline{post-conditions}} \dots

        not exactly about determining output \dots

        nor about program equivalence
        
        \pause
        \bigskip
        \bigskip
        \begin{center}
                \fbox{
                        Do we have the \alert{\underline{right semantics}}
                        for solving them ?
                }
        \end{center}
\end{frame}

\section{Key Points}

\begin{frame}{Key Points}

        Focussed on \alert{\underline{output properties}} and less on outputs
        themselves

        Centred around a \alert{\underline{logic}} 
        {\small (for reasoning about these
        properties)}

        Semantic rules are thus more \alert{\underline{logic oriented}}

        \bigskip
        \pause
        Good for \alert{\underline{program correctness}} 
        {\small (recall `algorithms and complexity')}
\end{frame}

\begin{frame}{Hoare Triples}

        Axiomatic semantics essentially about (dis)proving
        \[
                \{ \Phi \} \, \prog{p} \, \{ \Psi \}
        \]
        "If $\Phi$ holds at the input then $\Psi$ holds at the output"

        \bigskip
        \begin{block}{Examples}
                \begin{itemize}
                        \item $\{ \prog{tt} \} \, \prog{p} \, \{ \prog{x \geq 2} \}$
                        \item $\{ \prog{x =n \wedge y = 1} \} \, \prog{p} \, \{ \prog{y = n!} \}$
                        \item \dots
                \end{itemize}
        \end{block}
\end{frame}

\begin{frame}{Meaning of Hoare Triples}

        Can we state \alert{\underline{mathematically}} what a Hoare triple really means ?

        \pause
        \bigskip
        Question rooted on what a program means {\small (recall our lectures)}

        \dots \, and of course on the \alert{\underline{choice of a logic}} for properties

        \pause
        \bigskip
        Right choice often not obvious \, \dots
\end{frame}

\begin{frame}{The Choice}
        
        Often varies depending on the problem at hand

        \dots \, but typically the case that $\Phi$ corresponds to a subset
        \[
                \sem{\Phi} \subseteq \mathrm{State}_\bot
        \]
        {\small (`the elements of $\mathrm{State}_\bot$ at which $\Phi$ holds')}

        \pause
        \bigskip
        Scientists typically fix on the well-established \alert{\underline{first-order-logic}}

        \dots \, which however brings its own set of problems

\end{frame}

\begin{frame}{Meaning of Hoare Triples}
        \[
                \{ \Phi \} \, \prog{p} \, \{ \Psi \} \qquad \text{ means } \qquad 
                \Big ( \, x \in \sem{\Phi} \Longrightarrow \sem{\prog{p}}(x) \in \sem{\Psi} \, \Big )
        \]

        \pause
        \bigskip
        \bigskip
        Remarkably note the following equivalence
        \[
                \Big ( \, x \in \sem{\Phi} \Longrightarrow \sem{\prog{p}}(x) \in \sem{\Psi} \, \Big )
                \quad
                \text{ iff }
                \quad 
                \sem{\Phi} \subseteq \sem{\prog{p}}^{-1}(\sem{\psi})
        \]
        It is at the root of a rich theory of
        \begin{center}
                \fbox{
                        `backward transformations' known as \alert{\underline{predicate transformers}}
                }
        \end{center}
\end{frame}

\begin{frame}{Liberals vs. Conservatives}

        In the sequel we will consider only \alert{\underline{liberal conditions}}

        \dots \, \ie\ every predicate $\Phi$ will have $\bot \in \sem{\Phi}$

        \bigskip
        Entails that we are working only with \alert{\underline{partial correctness}}

        \dots \, \ie\ no predicate enforces termination
\end{frame}

\begin{frame}{Exercises}

        Argue informally whether the triples below hold
        \\[7pt]
        \begin{itemize}
                \item $\{ \prog{tt} \} \> \> \prog{{\blue while} \> \> tt \> \> skip} \> \> \{ \prog{ff} \}$
                        \\[7pt]
                \item $\{ \prog{tt} \} \> \> \prog{{\blue if} \>  b \> {\blue then} \> \> 
                        x:=2 \> \> {\blue else} \> \> x:= 3} \> \> \{ \prog{x \geq 2} \}$
                        \\[7pt]
                \item $\{ \prog{x = a \wedge y = b} \} \> \>
                        \prog{x:= y \> {\blue ;} \> y := x} \> \> \{ \prog{ x = b \wedge y = a }\}$
                        \\[7pt]
                \item $\{ \prog{x = a \wedge y = b} \} \> \>
                        \prog{aux := x \> {\blue ;} \> 
                        x:= y \> {\blue ;} \> y := aux} \> \> \{ \prog{ x = b \wedge y = a }\}$
                        \\[7pt]
                \item $\{ \prog{x =n \wedge y = 1} \} \> \> \prog{fact} \> \> \{ \prog{y = n!} \}$
        \end{itemize}
\end{frame}

\section{Weakest Precondition Semantics}

\begin{frame}{What and Why}

        Focus is on deriving the \alert{\underline{weakest}} condition $\Phi$ such that
        \[
                \{ \Phi \} \> \prog{p} \> \{ \Psi \}   \hspace{1cm}
                \Big ( \> \text{iff } \sem{\Phi} \subseteq \sem{\prog{p}}^{-1}(\sem{\Psi}) \> \Big )
        \]
        \pause
        $\Phi$ `weaker' (\ie\ less restrictive) than $\Phi'$ means $\sem{\Phi} \supseteq \sem{\Phi'}$

        \pause
        \bigskip
        \bigskip
        \begin{minipage}[0.3\textheight]{\textwidth}
                \begin{columns}[c]
                \begin{column}{0.7\textwidth}
                        To \alert{\underline{understand}} a program amounts to knowing the weakest
                        precondition that ensures a given postcondition
                \end{column}
                \begin{column}{0.22\textwidth}
                        \includegraphics[scale=1.1]{images/Dijkstra.jpg}
                \end{column}
                \end{columns}
       \end{minipage}
\end{frame}

\begin{frame}{The Semantics}
        \begin{align*}
                \mathrm{wp}\,(\prog{x : = e}, \Phi) & = \Phi[\prog{e}/\prog{x}] \\[5pt]
                %
                \mathrm{wp}\,(\prog{p} \> {\blue ;} \> \prog{q}, \Phi) & 
                = \mathrm{wp} \, (\prog{p}, \mathrm{wp} \, (\prog{q}, \Phi))  \\[5pt]
                %
                \mathrm{wp}\,({\mathtt{{\blue if} \> b \> {\blue then} \> p \> {\blue else} \> q}}, \Phi)
                & 
                =
                \prog{b} \wedge \mathrm{wp}\, (\prog{p},\Phi) \, \vee \,
                \prog{\neg b} \wedge \mathrm{wp} \, (\prog{q}, \Phi)
                \\[5pt]
                \mathrm{wp} \, ({\mathtt{{\blue while} \> b \> {\blue do} \> \{ \> p \> \}} }, \Phi)
                & = \, \dots 
        \end{align*}
\end{frame}

\begin{frame}{Exercises}
        
        \alert{\underline{Calculate}} the weakest preconditions w.r.t. the following pairs
        \\[7pt]
        \begin{itemize}
                \item $(\prog{x := y}, \>  \prog{x} \geq 1)$
                        \\[7pt]
                \item $(\prog{{\blue if} \>  b \> {\blue then} \> \> 
                        x:=2 \> \> {\blue else} \> \> x:= 3}, \> \prog{x \geq 2})$
                        \\[7pt]
                \item $(\prog{x:= y \> {\blue ;} \> y := x}, \> \prog{ x = b \wedge y = a })$
                        \\[7pt]
                \item $(\prog{ aux := x {\blue ;} \> 
                        x:= y \> {\blue ;} \> y := aux}, \> \prog{ x = b \wedge y = a })$
        \end{itemize}
\end{frame}

\begin{frame}{The Semantics}
        \begin{align*}
                \mathrm{wp}\,(\prog{x : = e}, \Phi) & = \Phi[\prog{e}/\prog{x}] \\[5pt]
                %
                \mathrm{wp}\,(\prog{p} \> {\blue ;} \> \prog{q}, \Phi) & 
                = \mathrm{wp} \, (\prog{p}, \mathrm{wp} \, (\prog{q}, \Phi))  \\[5pt]
                %
                \mathrm{wp}\,({\mathtt{{\blue if} \> b \> {\blue then} \> p \> {\blue else} \> q}}, \Phi)
                & 
                =
                \prog{b} \wedge \mathrm{wp}\, (\prog{p},\Phi) \, \vee \,
                \prog{\neg b} \wedge \mathrm{wp} \, (\prog{q}, \Phi)
                \\[5pt]
                \mathrm{wp} \, ({\mathtt{{\blue while} \> b \> {\blue do} \> \{ \> p \> \}} }, \Phi)
                & = \bigwedge_{n \in \Nats} \Psi_n
        \end{align*}
        \pause
        \vspace{-0.5cm}
        \begin{align*}
                \Psi_0 & = \prog{tt} \\
                \Psi_{n+1} & = \neg \prog{b} \wedge \Phi \, \vee \, \prog{b} \wedge 
                \mathrm{wp} \, (\prog{p}, \Psi_n)
        \end{align*}
\end{frame}

\begin{frame}{Unfolding While-loops}
        \begin{align*}
         & \, \mathrm{wp} \, ({\mathtt{{\blue while} \> b \> {\blue do} \> \{ \> p \> \}} }, \Phi)
         %
         \\
         %
         & = \Psi_0
         & {\scriptsize \text{(trivial)}^\ast }
         %
         \\
         %
         & \, \wedge \neg \prog{b} \wedge \Phi \, \vee \, \prog{b} \wedge \mathrm{wp} \, (\prog{p},\Psi_0)
         & {\scriptsize \text{(\alert{\underline{terminates}} with $\Phi$ 
         \alert{\underline{or iterates once}} and then \alert{\underline{$^\ast$}})}^{\ast \ast} }
         %
         \\
         %
         & \, \wedge \neg \prog{b} \wedge \Phi \, \vee \, \prog{b} \wedge \mathrm{wp} \, (\prog{p},\Psi_1)
         & {\scriptsize \text{(\alert{\underline{terminates}} 
                         with $\Phi$  \alert{\underline{or iterates once}} 
         and then \alert{\underline{$^{\ast \ast}$}})} }
         %
         \\
         %
         & \, \wedge \, \dots
        \end{align*}
        
        \pause
        \alert{\underline{Infinitary}} formula tracks when the loop terminates

        \dots\ in which case it enforces $\Phi$

        Each conjunct $\Psi_{n+1}$ tracks up to $n$ iterations
\end{frame}

\begin{frame}{Unfolding While-loops (The Case of Divergence) }
        \begin{align*}
         & \, \mathrm{wp} \, ({\mathtt{{\blue while} \> \prog{tt} \> {\blue do} \> \{ \> p \> \}} }, \Phi)
         %
         \\
         %
         & = \Psi_0 \qquad ( = \prog{tt} )
         & % 
         %
         \\
         %
         & \, \wedge \neg \prog{tt} \wedge \Phi \, \vee \, \prog{tt} \wedge \mathrm{wp} \, (\prog{p},\prog{tt})
         \qquad ( = \prog{ tt} )
         & % 
         %
         \\
         %
         & \, \wedge \neg \prog{tt} \wedge \Phi \, \vee \, \prog{tt} \wedge \mathrm{wp} \, (\prog{p},\prog{tt})
         \qquad ( = \prog{ tt } )
         &
         %
         \\
         %
         & \, \wedge \, \dots
         %
         \\
         & \, =  \prog{tt}
        \end{align*}
\end{frame}

\begin{frame}{Exercises}

        Prove that the following equations hold
        \\[10pt]
        \begin{itemize}
                \item $\mathrm{wp} \, (\prog{p}, \prog{tt}) = \prog{tt}$
                \\[10pt]
                \item $\mathrm{wp} \, (\prog{p}, \Phi \wedge \Psi )
                = \mathrm{wp} \, (\prog{p}, \Phi) \wedge 
                \mathrm{wp} \, (\prog{p}, \Psi )$
                \\[10pt]
                \item $\mathrm{wp} \, (\prog{p}, \bigwedge_{i \in I} \Phi_i )
                = \bigwedge_{i \in I} \mathrm{wp} \, (\prog{p}, \Phi_i)$
        \end{itemize}

\end{frame}

\begin{frame}{Pre-condition and Denotational Semantics}

        \begin{theorem}
                $\sem{\mathrm{wp} \, (\prog{p}, \Phi)} = \sem{\prog{p}}^{-1}(\sem{\Phi})$
        \end{theorem}

        \begin{proof}
                By \alert{\underline{induction}}. Case of while-loops
                proved neatly via domain theory
        \end{proof}

        \bigskip
        \begin{corollary}
                $\sem{\prog{p}} =
                \sem{\prog{q}} \Longrightarrow \forall \Phi. \, \mathrm{wp} \, (\prog{p}, \Phi)
                \equiv \mathrm{wp} \, (\prog{q}, \Phi)$
        \end{corollary}
\end{frame}

\begin{frame}{Expressivity Matters}

Is it true that $\Big ( \forall \Phi. \, \mathrm{wp}(\prog{p}, \Phi)
\equiv \mathrm{wp} \,(\prog{q}, \Phi) \Big ) \Longrightarrow \sem{\prog{p}} =
\sem{\prog{q}}$ ? 

\pause
\bigskip
Well \dots
\vspace{-0.3cm}
\begin{align*}
        & \, \forall \Phi. \, \mathrm{wp} \, (\prog{p}, \Phi) \equiv \mathrm{wp} \, (\prog{q}, \Phi)
        \\
        & \Longrightarrow 
        \forall \Phi. \, \sem{\mathrm{wp} \, (\prog{p}, \Phi)} = \sem{\mathrm{wp} \, (\prog{q}, \Phi)}
        \\
        & \Longrightarrow 
        \forall \Phi. \, \sem{\prog{p}}^{-1}(\sem{\Phi}) = \sem{\prog{q}}^{-1}(\sem{\Phi}) 
        &
        \\
        & \alert{\Longrightarrow} 
        \sem{\prog{p}} = \sem{\prog{q}} 
\end{align*}
\pause
\vspace{-0.5cm}
\begin{block}{A counter-example (the simplest grammar of propositions) }
        \[
                \prog{b} :: = \prog{tt} \mid \prog{\neg b} \mid \prog{b \vee b}
                \mid \bigwedge \prog{b}
        \]
        Calculate all possible interpretations $\sem{\prog{b}}$
\end{block}
\end{frame}

\begin{frame}{From Weakest Pre-conditions to Hoare Triples}

        We wish to prove the validity of Hoare triples

        \dots \, like in `algorithms and complexity'

        \bigskip
        For this we use a simple calculus from the semantics

        \dots \, with merely one rule
        \[
                \infer{\vdash \{ \Phi \} \, \prog{p} \, \{ \Psi \}}
                { \vdash \Phi \rightarrow \mathrm{wp} \, (\prog{p}, \Psi) }
        \]
\end{frame}

\begin{frame}{Soundness}

        Is our calculus correct ?

        \dots \, \ie\  $\vdash \{ \Phi \} \, \prog{p} \, \{ \Psi \} \Longrightarrow
        \sem{\Phi} \subseteq \sem{\prog{p}}^{-1}(\sem{\Psi})$

        \pause
        \bigskip
        Yes, and the proof is super easy !!
\end{frame}

\begin{frame}{Completeness}

        Is our calculus complete ?

        \dots \, \ie\  $\sem{\Phi} \subseteq \sem{\prog{p}}^{-1}(\sem{\Psi}) \Longrightarrow
        \> 
        \vdash \{ \Phi \} \, \prog{p} \, \{ \Psi \}$

        \pause
        \bigskip
        Well, it depends on whether the logic is complete

        \dots \, \ie\ $\sem{\Phi_1} \subseteq \sem{\Phi_2} \Longrightarrow \> \vdash \Phi_1 \rightarrow \Phi_2$
        for all formulae $\Phi_1, \Phi_2$

\end{frame}

\section{Hoare Calculus}

\begin{frame}{Motivation}

        Let us now try to establish
        \[
                \vdash \{ \prog{x =n \wedge y = 1} \} \> \> \prog{fact} \> \>
                \{ \prog{y = n!} \}
        \]
        ${\small \Big ( \prog{fact} = \prog{{\blue while} \> x > 0 \> \{ y :=
        x \times y \> {\blue ;} \> x := x - 1 \} } \Big )}$

        \pause
        \bigskip
        \bigskip
        \textbf{Still hard ? }

        The calculus is for obtaining \alert{\underline{weakest preconditions}}

        \dots \, and less about proving Hoare triples

        \dots \, besides infinite conjunctions are hard !
\end{frame}

\begin{frame}{Back to Old Friends}
        \[
                \infer{ \vdash \{ \Phi \} \> \prog{skip} \> \{ \Phi \} }{}
                \hspace{2.5cm}
                \infer{ \vdash \{ \Phi[\prog{e}/\prog{x}] \} \> \prog{ x : = e } \>
                \{ \Phi \} }{}
        \]
        \\[5pt]
        \[
                \infer { \vdash \{ \Phi \} \> \prog{p \> {\blue ;} \> q} \> \{ \Xi \} }{
                        \vdash \{ \Phi \} \> \prog{p} \> \{ \Psi \}
                        \qquad
                        \vdash \{ \Psi \} \> \prog{q} \> \{ \Xi \}
                }
                \hspace{1cm}
                \infer{
                        \vdash \{ \Phi \} \> \prog{{\blue while}} \> \prog{b} \>
                        \prog{p} \> 
                        \{ \neg \prog{b} \wedge \Phi \}
                }{
                        \vdash
                        \{ \Phi \wedge \prog{b} \} \> \prog{p} \> \{ \Phi \}
                }
        \]
        \\[5pt]
        \[
                \infer {
                        \vdash \{ \Phi \} \> \prog{{\blue if} \> b}
                        \> \prog{{\blue then}} \>
                        \prog{p} \> \prog{{\blue else}} \> \prog{q}
                        \> \{ \Psi \}
                }{
                        \vdash \{ \prog{b} \wedge \Phi \} \> \prog{p} \> \{ \Psi \}
                        \qquad
                        \vdash \{ \prog{\neg b} \wedge \Phi \} \> \prog{q} \> \{ \Psi \}
                }
        \]
        \\[5pt]
        \[
                \infer{ \vdash \{ \Phi \} \> \prog{p} \> \{ \Psi \} }{
                        \vdash \Phi \rightarrow \Psi
                        \quad
                        \vdash \{ \Psi \} \> \prog{p} \> \{ \Xi \}
                        \quad
                        \vdash \Xi \rightarrow \Omega
                }
        \]
\end{frame}
\bibliographystyle{amsalpha}
\bibliography{axiomatic}

\end{document}

