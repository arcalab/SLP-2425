\documentclass{beamer}
\usepackage{etex} % fixes new-dimension error
\usepackage{lmodern}
\usepackage[T1]{fontenc}
\usetheme{metropolis}
\metroset{block=fill}
%\usetheme{Boadilla}

\setbeamertemplate{footline}
{
  \leavevmode%
  \hbox{%
  \begin{beamercolorbox}[wd=.4\paperwidth,ht=2.25ex,dp=1ex,center]{author in head/foot}%
    \usebeamerfont{author in head/foot}\insertshortauthor
  \end{beamercolorbox}%
  \begin{beamercolorbox}[wd=.5\paperwidth,ht=2.25ex,dp=1ex,center]{title in head/foot}%
    \usebeamerfont{title in head/foot}\insertsection
  \end{beamercolorbox}%
  \begin{beamercolorbox}[wd=.1\paperwidth,ht=2.25ex,dp=1ex,right]{date in head/foot}%
    \insertframenumber{} / \inserttotalframenumber\hspace*{2ex} 
  \end{beamercolorbox}}%
  \vskip0pt%
}
\makeatother
\setbeamertemplate{navigation symbols}{}
%----------------------------------------------------------------------------
\usepackage{graphicx,amsmath}
\usepackage{stmaryrd} % cf. interleave
\usepackage{booktabs}
\usepackage{amscd}
\usepackage{multicol}
\usepackage[absolute,overlay]{textpos}
\usepackage{alltt}
\usepackage{proof}
\usepackage{listings}
\usepackage[all]{xy}
\input{macros}
%------ using pstricks (rnode etc) ------------------------------------------
\usepackage{pstricks,pst-node,pst-text,pst-3d}

% ------ using color ---------------------------------------------------------
\newrgbcolor{goldenrod}{.80392 .60784 .11373}
\newrgbcolor{darkgoldenrod}{.5451 .39608 .03137}
\newrgbcolor{brown}{.15 .15 .15}
\newrgbcolor{darkolivegreen}{.33333 .41961 .18431}
\def\gold#1{{\goldenrod #1}}
\def\dgold#1{{\alert{#1}}}
\def\dkb#1{{\blue #1}}
\def\tdkb#1{\textbf{\darkblue #1}}
\def\gre#1{{\darkolivegreen #1}}
\def\gry#1{{\gray #1}}
\def\rdb#1{{\red #1}}
%----------------------------------------------------------------------------

\AtBeginSection[]
{
    \begin{frame}
        \frametitle{Table of Contents}
        \tableofcontents[currentsection]
    \end{frame}
}

\author[Renato Neves]{Renato Neves}

% logos of institutions
\titlegraphic{
  \begin{textblock*}{5cm}(6.5cm,7.4cm)
     \includegraphics[scale=0.06]{images/uminho.png}\hspace*{.85cm}~%
  \end{textblock*}
  \begin{textblock*}{5cm}(9.2cm,7.45cm)
    \includegraphics[scale=0.50]{images/haslab.pdf}
  \end{textblock*}
}

% No date
\date{}

\begin{document}

\title{Axiomatic Semantics}

\frame[plain]{\titlepage}


\begin{frame}{Semantics for Every Season}

        \makebox[\linewidth][c]{%
        \begin{tabular}{ l l }
        Operational semantics & How a program operates
        \\
        Denotational semantics & What a program is 
        \\
        \alert{\underline{Axiomatic semantics}} 
                               & Which logical properties it satisfies
        \end{tabular}
        }

\end{frame}

\section{Motivation}

\begin{frame}{A Brief Warm-up} 

        Solve the following exercises via your favorite semantics  
        \smallskip
        \begin{itemize}
                \item Calculate the output of $\prog{ x := 1 \> {\blue ;} \> x
                        := 2}$
                %
                \\[15pt]
                %
                \item Show that the following program outputs a state with
                $\prog{x} \geq 2$
                \begin{center}
                        $\prog{{\blue if} \> \> x = 1 \> \> {\blue then } \> \>
                        x := 2 \> \> {\blue else} \> \> x := 3}$
                %
                \\[15pt]
                %
                \end{center}
                %
                \item Show that the following program is the factorial
                function
                \begin{center}
                        $\prog{{\blue while} \> x > 0 \> \{ y := x \times y
                        \> {\blue ;} \> x := x - 1 \} }$
                \end{center}
        \end{itemize} 

        \pause
        \bigskip
        \textbf{Hard ?}
\end{frame}

\begin{frame}{The Right Tools}

        Two last exercises were about \alert{\underline{post-conditions}} \dots

        not exactly about determining output \dots

        nor about program equivalence
        
        \pause
        \bigskip
        \bigskip
        \begin{center}
                \fbox{
                        Do we have the \alert{\underline{right semantics}}
                        for solving them ?
                }
        \end{center}
\end{frame}

\section{Key Points}

\begin{frame}{Key Points}

        Focussed on \alert{\underline{output properties}} and less on outputs
        themselves

        Centred around a \alert{\underline{logic}} 
        {\small (for reasoning about these
        properties)}

        Semantic rules are thus more \alert{\underline{logic oriented}}

        \bigskip
        \pause
        Good for \alert{\underline{program correctness}} 
        {\small (recall `algorithms and complexity')}
\end{frame}

\begin{frame}{Hoare Triples}

        Axiomatic semantics essentially about (dis)proving
        \[
                \{ \Phi \} \, \prog{p} \, \{ \Psi \}
        \]
        "If $\Phi$ holds at the input then $\Psi$ holds at the output"

        \bigskip
        \begin{block}{Examples}
                \begin{itemize}
                        \item $\{ \prog{tt} \} \, \prog{p} \, \{ \prog{x \geq 2} \}$
                        \item $\{ \prog{x =n \wedge y = 1} \} \, \prog{p} \, \{ \prog{y = n!} \}$
                        \item \dots
                \end{itemize}
        \end{block}
\end{frame}

\begin{frame}{Meaning of Hoare Triples}

        Can we state \alert{\underline{mathematically}} what a Hoare triple really means ?

        \pause
        \bigskip
        Question rooted on what a program means {\small (recall our lectures)}

        \dots \, and of course on the \alert{\underline{choice of a logic}} for properties

        \pause
        \bigskip
        Right choice often not obvious \, \dots
\end{frame}

\begin{frame}{The Choice}
        
        Often varies depending on the problem at hand

        \dots \, but typically the case that $\Phi$ corresponds to a subset
        \[
                \sem{\Phi} \subseteq \mathrm{State}_\bot
        \]
        {\small (`the elements of $\mathrm{State}_\bot$ at which $\Phi$ holds')}

        \pause
        \bigskip
        Scientists typically fix on the well-established \alert{\underline{first-order-logic}}

        \dots \, which however brings its own set of problems

\end{frame}

\begin{frame}{Meaning of Hoare Triples}
        \[
                \{ \Phi \} \, \prog{p} \, \{ \Psi \} \qquad \text{ means } \qquad 
                \Big ( \, x \in \sem{\Phi} \Longrightarrow \sem{\prog{p}}(x) \in \sem{\Psi} \, \Big )
        \]

        \pause
        \bigskip
        \bigskip
        Remarkably note the following equivalence
        \[
                \Big ( \, x \in \sem{\Phi} \Longrightarrow \sem{\prog{p}}(x) \in \sem{\Psi} \, \Big )
                \quad
                \text{ iff }
                \quad 
                \sem{\Phi} \subseteq \sem{\prog{p}}^{-1}(\sem{\psi})
        \]
        It is at the root of a rich theory of
        \begin{center}
                \fbox{
                        `backward transformations' known as \alert{\underline{predicate transformers}}
                }
        \end{center}
\end{frame}

\begin{frame}{Liberals vs. Conservatives}

        In the sequel we will consider only \alert{\underline{liberal conditions}}

        \dots \, \ie\ every proposition $\Phi$ will have $\bot \in \sem{\Phi}$

        \bigskip
        Entails that we are working only with \alert{\underline{partial correctness}}

        \dots \, \ie\ no proposition enforces termination
\end{frame}

\bibliographystyle{amsalpha}
\bibliography{axiomatic}

\end{document}

