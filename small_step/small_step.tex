\documentclass{beamer}
\usepackage{etex} % fixes new-dimension error
\usepackage{lmodern}
\usepackage[T1]{fontenc}
\usetheme{metropolis}
\metroset{block=fill}
%\usetheme{Boadilla}

\setbeamertemplate{footline}
{
  \leavevmode%
  \hbox{%
  \begin{beamercolorbox}[wd=.4\paperwidth,ht=2.25ex,dp=1ex,center]{author in head/foot}%
    \usebeamerfont{author in head/foot}\insertshortauthor
  \end{beamercolorbox}%
  \begin{beamercolorbox}[wd=.5\paperwidth,ht=2.25ex,dp=1ex,center]{title in head/foot}%
    \usebeamerfont{title in head/foot}\insertsection
  \end{beamercolorbox}%
  \begin{beamercolorbox}[wd=.1\paperwidth,ht=2.25ex,dp=1ex,right]{date in head/foot}%
    \insertframenumber{} / \inserttotalframenumber\hspace*{2ex} 
  \end{beamercolorbox}}%
  \vskip0pt%
}
\makeatother
\setbeamertemplate{navigation symbols}{}
%----------------------------------------------------------------------------
\usepackage{graphicx,amsmath}
\usepackage{stmaryrd} % cf. interleave
\usepackage{booktabs}
\usepackage{amscd}
\usepackage{multicol}
\usepackage[absolute,overlay]{textpos}
\usepackage{alltt}
\usepackage{proof}
\usepackage{listings}
\input{macros}
%------ using pstricks (rnode etc) ------------------------------------------
\usepackage{pstricks,pst-node,pst-text,pst-3d}

% ------ using color ---------------------------------------------------------
\newrgbcolor{goldenrod}{.80392 .60784 .11373}
\newrgbcolor{darkgoldenrod}{.5451 .39608 .03137}
\newrgbcolor{brown}{.15 .15 .15}
\newrgbcolor{darkolivegreen}{.33333 .41961 .18431}
\def\gold#1{{\goldenrod #1}}
\def\dgold#1{{\alert{#1}}}
\def\dkb#1{{\blue #1}}
\def\tdkb#1{\textbf{\darkblue #1}}
\def\gre#1{{\darkolivegreen #1}}
\def\gry#1{{\gray #1}}
\def\rdb#1{{\red #1}}
%----------------------------------------------------------------------------

\AtBeginSection[]
{
    \begin{frame}
        \frametitle{Table of Contents}
        \tableofcontents[currentsection]
    \end{frame}
}

\author[Renato Neves]{Renato Neves}

% logos of institutions
\titlegraphic{
  \begin{textblock*}{5cm}(6.5cm,7.4cm)
     \includegraphics[scale=0.06]{images/uminho.png}\hspace*{.85cm}~%
  \end{textblock*}
  \begin{textblock*}{5cm}(9.2cm,7.45cm)
    \includegraphics[scale=0.50]{images/haslab.pdf}
  \end{textblock*}
}

% No date
\date{}

\begin{document}

\title{Small-step Semantics}

\frame[plain]{\titlepage}

\section{Outline}

\begin{frame}{Semantics for every season}

        \hspace*{+5pt}\makebox[\linewidth][c]{%
        \begin{tabular}{ l l }
        \alert{\underline{Operational semantics}} & How a program operates
        \\
        Denotational semantics & What a program is 
        \\
        Axiomatic semantics & Which logical properties a program satisfies
        \end{tabular}
        }

\end{frame}


\begin{frame}{Small-step operational semantics}

        \alert{\underline{How}} a program operates \alert{\underline{step-by-step}}

        \begin{example}
                $\pv{\prog{x := 1 ; x := x +1}}{v} \longrightarrow
                \pv{\prog{x := x + 1}}{1} \longrightarrow 2$
        \end{example}

        A machine with an \alert{\underline{'evaluation stack'}} that is
        processed at each step 
\end{frame}

\begin{frame}{Its uses}

        Describes how a program operates step-by-step

        Describes \alert{\underline{evaluation techniques}} (\eg\
        short-circuiting)

        A basis for tracing/debugging

        Foundations of concurrency, complexity, \dots

        Rich notions of equivalence

        \dots

        \medskip
        \pause
        Thus an essential tool for \alert{\underline{understanding}} a
        programming language

\end{frame}

\section{First steps}

\begin{frame}{A propositional language}

        \begin{block}{\vspace*{-3.5ex}}
        \[
                \prog{b} ::= \prog{x} \mid \prog{b} \wedge \prog{b} \mid \neg \prog{b}
        \]
        \end{block} 

        \bigskip
        \centering
        Every $\prog{x}$ is a \alert{\underline{proposition}} (\ie\ it has
        either value $\prog{tt}$ or $\prog{ff}$)

        \pause
        \vfill
        \centering
        \fbox{ Can we provide a small-step semantics to this language ? }
\end{frame}

\begin{frame}{Key points}

        Uses a \alert{\underline{memory}} $\sigma : X \to
        \mathrm{Bool}$
        that assigns to every proposition $\prog{x}$ its truth-value
        $\sigma(\prog{x})$

        \bigskip
        A term $\prog{b}$ is evaluated step-by-step 
        until a truth-value $v$ is reached
        \[
                \bullet \longrightarrow \bullet \longrightarrow \cdots 
                \longrightarrow 
                \bullet \longrightarrow \bullet \longrightarrow v
        \]
        

        Focus is on \alert{\underline{the next step}} (of the evaluation)
\end{frame}

\begin{frame}{The semantics -- a universe of laws}
        \[
                \infer[(\text{var})]{\langle \mathtt{x}, \sigma \rangle 
                \longrightarrow \sigma(\mathtt{x})}{} \qquad \qquad
                \qquad
                %
                \infer[(\text{neg$_1$})]{\langle \neg \prog{b}, \sigma \rangle 
                \longrightarrow \neg v}{
                   \pv{\prog{b}}{\sigma} \longrightarrow
                   v
                }
        \] 
        \pause
        \vspace{0.1cm}        
        \[
                \infer[(\text{neg$_2$})]{
                \pv{\neg \prog{b}}{\sigma} 
                \longrightarrow \pv{\neg \prog{b'}}{\sigma'}
                }{
                \pv{\prog{b}}{\sigma} 
                \longrightarrow \pv{\prog{b'}}{\sigma'}
                } \qquad \qquad
                %
                \infer[(\text{and$_1$})]{
                        \pv{\prog{b_1} \wedge \prog{b_2} }{\sigma}
                \longrightarrow \prog{ff}}{
                   \pv{\prog{b_1}}{\sigma} \longrightarrow
                   \prog{ff}
                }
        \] 
        \vspace{0.1cm}
        \[
                %
                \hspace{-0.3cm}
                \infer[(\text{and$_2$})]{
                %
                \pv{ \prog{b_1} \wedge \prog{b_2} }{ \sigma }
                \longrightarrow 
                \pv{ \prog{b_2} }{\sigma}  }
                {
                   \pv{\prog{b_1}}{\sigma} \longrightarrow
                   \prog{tt}
                }
                \qquad 
                \infer[(\text{and$_3$})]{
                        \pv{\prog{b_1} \wedge \prog{b_2} }{\sigma}
                \longrightarrow \pv{\prog{b_1'} \wedge \prog{b_2}}{\sigma'} }{
                   \pv{\prog{b_1}}{\sigma} \longrightarrow
                   \pv{\prog{b_1'}}{\sigma'}
                }
        \] 
        \vfill
\end{frame}

\begin{frame}{An example}

        $\neg \neg \prog{x} \longrightarrow\ ?$

        \pause
        \centering
        \[
                \infer[(\text{neg$_1$})]{
                        \pv{\neg \neg \prog{x}}{\sigma}
                        \longrightarrow\ \neg \neg \sigma(\prog{x})
                }{
                        \infer[(\text{neg$_1$})]{
                                \pv{\neg \prog{x}}{\sigma}
                                \longrightarrow\ \neg \sigma(\prog{x})
                        }{
                                \infer[(\text{var})]{ 
                                \pv{\prog{x}}{\sigma} \longrightarrow\  \sigma(\prog{x})
                                }{
                                }
                        }
                }
        \]
\end{frame}

\begin{frame}{Another example}

        $(\prog{x} \wedge \prog{b_1}) \wedge \prog{b_2} \longrightarrow\ ?$

        \centering
        \bigskip
        If $\sigma(\prog{x}) = \prog{ff}$:
        \smallskip
        \[
                \infer[(\text{and$_1$})]{
                        \pv{(\prog{x} \wedge \prog{b_1}) \wedge
                        \prog{b_2}}{\sigma} \longrightarrow\ \prog{ff} 
                }{
                        \infer[(\text{and$_1$})]{
                                \pv{\prog{x} \wedge \prog{b_1}}{\sigma} 
                                \longrightarrow\ \prog{ff}
                        }{
                                \infer[(\text{var})]{ 
                                        \pv{\prog{x}}{\sigma} \longrightarrow\  \prog{ff}
                                }{
                                }
                        }
                }
        \]

\end{frame}

\begin{frame}{Yet another example}

        $(\prog{x} \wedge \prog{b_1}) \wedge \prog{b_2} \longrightarrow\ ?$

        \centering
        \bigskip
        If $\sigma(\prog{x}) = \prog{tt}$:
        \smallskip
        \[
                \infer[(\text{and$_3$})]{
                        \pv{(\prog{x} \wedge \prog{b_1}) \wedge
                        \prog{b_2}}{\sigma} \longrightarrow\ 
                        \pv{\prog{b_1} \wedge \prog{b_2}}{\sigma}
                }{
                        \infer[(\text{and$_2$})]{
                                \pv{\prog{x} \wedge \prog{b_1}}{\sigma} 
                                \longrightarrow\ \pv{\prog{b_1}}{\sigma}
                        }{
                                \infer[(\text{var})]{ 
                                        \pv{\prog{x}}{\sigma} \longrightarrow\  \prog{tt}
                                }{
                                }
                        }
                }
        \]

\end{frame}

\begin{frame}{Now you try !}

        $\prog{x} \wedge \neg \prog{x} \longrightarrow\ ?$

        $\neg (\neg \prog{x} \wedge \neg \prog{y}) \longrightarrow\ ?$

        \bigskip
        Provide semantics to the Boolean implication $\prog{b} \Rightarrow
        \prog{b}$
\end{frame}

\begin{frame}{From one step to many \dots}
        
        One often is uninterested on the next step \dots

        \dots and rather on the \alert{\underline{output}} (that the sequence of
        steps leads to)

        \pause
        \bigskip
        This \alert{\underline{multi-step transition}}  $\longrightarrow^n$
        is defined by the rules 
        \smallskip
        \[
                \infer[\text{(stp)}]{ 
                        \pv{\prog{b}}{\sigma} \longrightarrow^1 v 
                }{
                        \pv{\prog{b}}{\sigma} \longrightarrow v 
                }
                \hspace{1cm}
                \infer[\text{(nxt)}]{ 
                        \pv{\prog{b}}{\sigma} \longrightarrow^{n+1} v 
                }{
                        \pv{\prog{b}}{\sigma} \longrightarrow 
                        \pv{\prog{b'}}{\sigma'} 
                        \qquad
                        \pv{\prog{b'}}{\sigma'} \longrightarrow^n v
                }
        \]
\end{frame}

\begin{frame}{What's next ?}

        Fine, we have an operational semantics; so what ?

        \pause
        \bigskip
        We can now prove \alert{\underline{cool properties}} about our language
        !!

        \smallskip
        \begin{example}[Termination]
                It is always the case that $\pv{\prog{b}}{\sigma}
                \longrightarrow^n v$ for some $v$ and $n$
        \end{example}
%        \smallskip
%        \begin{example}[Determinism]
%                If $\pv{\prog{b}}{\sigma} \longrightarrow^\star v$ and
%                $\pv{\prog{b}}{\sigma} \longrightarrow^\star v'$ then $v = v'$
%        \end{example}
\end{frame}

\begin{frame}{Exercise 1}

        Define a `\alert{\underline{complexity}} function'
        \begin{align*}
                \mathrm{compl}(\prog{x}) & = 1  \\
                \mathrm{compl}(\prog{\neg b}) & = \mathrm{compl}(\prog{b}) \\
                \mathrm{compl}(\prog{b}_1 \wedge \prog{b}_2) & =
                \mathrm{compl}(\prog{b}_1) + \mathrm{compl}(\prog{b}_2)
        \end{align*}
        
        \medskip
        Show by \alert{\underline{induction}} that $\mathrm{compl}(\prog{b})
        \geq 1$ for every $\prog{b}$
\end{frame}

\begin{frame}{Exercise 2}

        Show by \alert{\underline{induction}} the following 
        implication
        \[
                \text{If } \pv{\prog{b}}{\sigma} \longrightarrow \pv{\prog{b}'}{\sigma'} 
                \text{ then }
                \mathrm{compl}(\prog{b}) > \mathrm{compl}(\prog{b}')
       \]
\end{frame}

\begin{frame}{Exercise 3}
        Show by \alert{\underline{induction}} the following 
        implication
        \[
                \text{If } \pv{\prog{b}}{\sigma} \longrightarrow^n v 
                \text{ then }
                \mathrm{compl}(\prog{b}) \geq n 
       \]
\end{frame}

\begin{frame}{When the number of steps does not matter \dots}

        One often is uninterested on the number of steps \dots

        \dots and rather just on the \alert{\underline{output}}

        \pause
        \bigskip
        This \alert{\underline{multi-step transition}}  $\longrightarrow^\star$
        is defined by the rules 
        \smallskip
        \[
                \infer[\text{(stp)}]{ 
                        \pv{\prog{b}}{\sigma} \longrightarrow^\star v 
                }{
                        \pv{\prog{b}}{\sigma} \longrightarrow v 
                }
                \hspace{1cm}
                \infer[\text{(nxt)}]{ 
                        \pv{\prog{b}}{\sigma} \longrightarrow^\star v 
                }{
                        \pv{\prog{b}}{\sigma} \longrightarrow 
                        \pv{\prog{b'}}{\sigma'} 
                        \qquad
                        \pv{\prog{b'}}{\sigma'} \longrightarrow^\star v
                }
        \]

\end{frame}

\begin{frame}{Exercise 4}
        Show by \alert{\underline{induction}} the following 
        equivalence
        \[
                \pv{\prog{b}}{\sigma} \longrightarrow^n v\
                 (\text{for some } n)
                \text{ iff }
                \pv{\prog{b}}{\sigma} \longrightarrow^\star v 
       \]
\end{frame}

\section{Second steps: small-step for a while-language}

\begin{frame}{A simple while-language}

        \vspace{0.7cm}
	\begin{block}{Arithmetic expressions}
	$e ::=  \mathtt{n}  \mid \mathtt{e \cdot e}
        \mid  \mathtt{x}  \mid \mathtt{e + e}$
	\end{block}

	\vspace{0.7cm}
	\begin{block}{Programs}
        $\prog{p} ::= \mathtt{x} := \mathtt{e} \mid
	\mathtt{p \> {\blue ;} \> q} \mid
	\mathtt{{\blue if} \> b \> {\blue then} \> p \> {\blue else} \> q} \mid
	\mathtt{{\blue while} \> b \> {\blue do} \> \{ \> p \> \}}$
	\end{block}

\end{frame}

\begin{frame}{A while-language and its semantics}
        \[
                \infer[(\text{asg})]{
                        \langle \mathtt{x := e}, \sigma \rangle \longrightarrow
                        \sigma[v / \mathtt{x}]
                }{
                       \langle \mathtt{e}, \sigma \rangle \longrightarrow^\star v
                } \hspace{1cm}
                \infer[(\text{seq$_1$})]{
                        \langle \mathtt{p \> {\blue ;} \> q}, \sigma \rangle \longrightarrow 
                        \pv{\prog{q}}{\sigma'}
                }{
                        \langle \mathtt{p}, \sigma \rangle \longrightarrow \sigma'
                }
        \]
        \vspace{0.001cm}
        \[
                \hspace{-0.5cm}
                \infer[(\text{seq$_2$})]{
                        \langle \mathtt{p \> {\blue ;} \> q}, \sigma \rangle \longrightarrow 
                        \pv{\prog{p' \> {\blue ;} \> q}}{\sigma'}
                }{
                        \langle \mathtt{p}, \sigma \rangle 
                        \longrightarrow \pv{\prog{p'}}{\sigma'}
                }
                \hspace{0.5cm}
                \infer[(\text{if$_1$})]{
                        \langle \mathtt{{\blue if} \> b \> {\blue then} \> \> 
                        p \> {\blue else} \> q}, 
                        \sigma \rangle \longrightarrow \pv{\prog{p}}{\sigma}
                }{
                        \langle \mathtt{b}, \sigma \rangle \longrightarrow^\star \mathtt{tt} 
                } 
        \]
        \vspace{0.001cm}
        \[      \hspace{-0.9cm}
                \infer[(\text{if$_2$})]{
                        \langle \mathtt{{\blue if} \> b \> {\blue then} \> \> 
                        p \> {\blue else} \> q}, 
                        \sigma \rangle \longrightarrow \pv{\prog{q}}{\sigma}
                }{
                        \langle \mathtt{b}, \sigma \rangle \longrightarrow^\star \mathtt{ff} 
                } 
                \hspace{0.4cm}
                \infer[(\text{wh$_2$})]{
                        \langle 
                        \mathtt{{\blue while} \> b \> {\blue do} \> \{ \> p \> \}}, \sigma \rangle
                        \longrightarrow^\star \sigma
                }{
                        \langle \mathtt{b}, \sigma \rangle \longrightarrow^\star \mathtt{ff}
                }
        \]\vspace{0.001cm}
        \[
                \infer[(\text{wh$_1$})]{
                        \langle \mathtt{{\blue while} \> b \> {\blue do} \> \{ \> p \> \}}, 
                        \sigma \rangle
                        \longrightarrow 
                        \pv{\prog{p} \> {\blue ;} \> 
                        \mathtt{{\blue while} \> b \> {\blue do} \> \{ \> p \> \}}}{
                        \sigma}
                }{
                        \pv{\prog{b}}{\sigma} \longrightarrow^\star \prog{tt} 
                }
        \]
\end{frame}

\begin{frame}{Exercises}

        \begin{enumerate}
                \item Define a semantics for arithmetic expressions

                \item Write down the sequence of steps that originates from
                        \[
                        \pv{\mathtt{{\blue while} \> \prog{tt} \> {\blue do} 
                        \> \{ \> x := x +1 \> \}}}{\sigma}
                        \]
                \item Conclude that the while-language does not enjoy the
                        termination property
        \end{enumerate}
\end{frame}
\end{document}
