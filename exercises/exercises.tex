\documentclass[a4paper, 11pt]{article}

%% packages
\usepackage{fullpage} % changes the margin
\usepackage{hyperref} % Links
\usepackage[utf8]{inputenc}
\usepackage{lmodern}
\usepackage{amsfonts}
\usepackage{amsthm}
\usepackage{amsmath}
\usepackage{amssymb}
\usepackage{stmaryrd}
\usepackage{proof}
\usepackage{braket}
\usepackage{graphicx}
\usepackage{tikz}
\usepackage{tikz-cd}
\usepackage{quantikz}
\usepackage[linewidth=1pt]{mdframed}
%%

%% environments
\theoremstyle{definition}
\newtheorem{definition}{Definition}
\newtheorem{examples}{Example}
\newtheorem{exercises}{Exercises}
\newtheorem{exercise}{Exercise}
\newtheorem{postulate}{Postulate}
\newtheorem{problem}{Problem}
%%

%% config
\date{}
\linespread{1.15}
\newcommand{\blue}[1]{\textcolor{blue}{#1}}
\input{macros}
%%

\begin{document}

\allowdisplaybreaks[2]
\title{\large{Semantics of Programming Languages 2024/25}
        \\ \large{University of Minho} 
}

\author{\small{Renato Neves} \small
(\href{mailto:nevrenato@di.uminho.pt}{nevrenato@di.uminho.pt})}
\maketitle

\section{Summary}

This document presents a list of exercises to help you prepare for the test.
It addresses the three styles of semantics that we have studied so far: namely
operational, denotational, and axiomatic semantics.

\section{Operational Semantics}

\subsection{Small-step}
        We started our lectures with the following very simple propositional
        language.
        \[
                \prog{b} ::= \prog{x} \mid \prog{b} \wedge \prog{b} \mid \neg \prog{b}
        \]
        Recall that $\prog{x}$ denotes a variable (from a given stock of
        variables).  Next we established the (small-step) operational semantics
        detailed in Figure~\ref{fig:small1}.
        %
        \begin{figure}[h]
        \begin{minipage}{1\textwidth}
        \begin{flalign*}
                \infer[(\text{var})]{\langle \mathtt{x}, \sigma \rangle 
                \longrightarrow \sigma(\mathtt{x})}{} \qquad \qquad
                %
                \hspace{2cm}
                %
                \infer[(\text{neg$_1$})]{\langle \neg \prog{b}, \sigma \rangle 
                \longrightarrow \neg v}{
                   \pv{\prog{b}}{\sigma} \longrightarrow
                   v
                }
        \end{flalign*}
        %%
        \\[-30pt]
        %%
        \begin{flalign*}
                \infer[(\text{neg$_2$})]{
                \pv{\neg \prog{b}}{\sigma} 
                \longrightarrow \pv{\neg \prog{b'}}{\sigma'}
                }{
                \pv{\prog{b}}{\sigma} 
                \longrightarrow \pv{\prog{b'}}{\sigma'}
                } 
                %
                \hspace{2cm}
                %
                \infer[(\text{and$_1$})]{
                        \pv{\prog{b_1} \wedge \prog{b_2} }{\sigma}
                \longrightarrow \prog{ff}}{
                   \pv{\prog{b_1}}{\sigma} \longrightarrow
                   \prog{ff}
                }
        \end{flalign*}
        %%
        \\[-30pt]
        %%
        \begin{flalign*}
                %
                \infer[(\text{and$_2$})]{
                %
                \pv{ \prog{b_1} \wedge \prog{b_2} }{ \sigma }
                \longrightarrow 
                \pv{ \prog{b_2} }{\sigma}  }
                {
                   \pv{\prog{b_1}}{\sigma} \longrightarrow
                   \prog{tt}
                }
                %
                \hspace{2cm}
                %
                \infer[(\text{and$_3$})]{
                        \pv{\prog{b_1} \wedge \prog{b_2} }{\sigma}
                \longrightarrow \pv{\prog{b_1'} \wedge \prog{b_2}}{\sigma'} }{
                   \pv{\prog{b_1}}{\sigma} \longrightarrow
                   \pv{\prog{b_1'}}{\sigma'}
                }
        \end{flalign*}
        \end{minipage}
        \label{small_prop}
        \caption{Small-step operational semantics for the propositional language}
        \label{fig:small1}
        \end{figure}

        \begin{problem}
                It makes sense to add an implication construct $\prog{b}
                \Rightarrow \prog{b}$ to the simple language. Which semantic
                rules would you add to Figure~\ref{fig:small1} in order to
                cover this new construct?
        \end{problem}

        In the lectures we have also defined the `complexity function'         
        \begin{align*}
                \mathrm{compl}(\prog{x}) & = 1  \\
                \mathrm{compl}(\prog{\neg b}) & = \mathrm{compl}(\prog{b}) \\
                \mathrm{compl}(\prog{b}_1 \wedge \prog{b}_2) & =
                \mathrm{compl}(\prog{b}_1) + \mathrm{compl}(\prog{b}_2)
        \end{align*}
        In a nutshell, it provides an upper bound to the number of steps a
        proposition needs to be fully evaluated.

        \begin{problem}
                Show by induction (on the syntactic structure of propositions)
                that $\mathrm{compl}(\prog{b}) \geq 1$ for every $\prog{b}$.
                Next, show by induction (over the depth of derivation trees)
                that the implication below holds for all propositions
                $\prog{b}$ and $\prog{b'}$.
                \[
                        \text{ If }\pv{\prog{b}}{\sigma} \longrightarrow
                        \pv{\prog{b'}}{\sigma}
                        \text{ then } \mathrm{compl}(\prog{b}) > \mathrm{compl}(\prog{b'})
                \]
        \end{problem}

        Let us consider now the grammars of arithmetic expressions and while-programs 
        that were studied throughout the lectures. They were defined as follows.
        \[
                \prog{e} ::=  \mathtt{n}  \mid \mathtt{e \cdot e}
                \mid  \mathtt{x}  \mid \mathtt{e + e}
        \]
        \[
                \prog{p} ::= \mathtt{x} := \mathtt{e} \mid
        	\mathtt{p \> \blue{;} \> p} \mid
        	\mathtt{\blue{if} \> b \> \blue{then} \> p \> \blue{else} \> p} \mid
        	\mathtt{\blue{while} \> b \> \blue{do} \> \{ \> p \> \}}
        \]
        \begin{problem}
                Remarkably in the lectures we did not see a small-step
                operational semantics for the arithmetic expressions. Can you
                define one such semantics now?
        \end{problem}

        What we did see however was a small-step operational semantics for the while-language,
        which is now detailed in Figure~\ref{fig:small2}.
        \begin{figure}[h]
        \begin{minipage}{1\textwidth}
        \begin{flalign*}
                \infer[(\text{asg})]{
                        \langle \mathtt{x := e}, \sigma \rangle \longrightarrow
                        \sigma[v / \mathtt{x}]
                }{
                       \langle \mathtt{e}, \sigma \rangle \longrightarrow^\star v
                } \hspace{2cm}
                \infer[(\text{seq$_1$})]{
                        \langle \mathtt{p \> \blue{;} \> q}, \sigma \rangle \longrightarrow 
                        \pv{\prog{q}}{\sigma'}
                }{
                        \langle \mathtt{p}, \sigma \rangle \longrightarrow \sigma'
                }
        \end{flalign*}
        %%
        \\[-30pt]
        %%
        \begin{flalign*}
                \infer[(\text{seq$_2$})]{
                        \langle \mathtt{p \> \blue{;} \> q}, \sigma \rangle \longrightarrow 
                        \pv{\prog{p' \> \blue{;} \> q}}{\sigma'}
                }{
                        \langle \mathtt{p}, \sigma \rangle 
                        \longrightarrow \pv{\prog{p'}}{\sigma'}
                }
                \hspace{2cm}
                \infer[(\text{if$_1$})]{
                        \langle \mathtt{\blue{if} \> b \> \blue{then} \> \> 
                        p \> \blue{else} \> q}, 
                        \sigma \rangle \longrightarrow \pv{\prog{p}}{\sigma}
                }{
                        \langle \mathtt{b}, \sigma \rangle \longrightarrow^\star \mathtt{tt} 
                } 
        \end{flalign*}
        %%
        \\[-30pt]
        %%
        \begin{flalign*}
                \infer[(\text{if$_2$})]{
                        \langle \mathtt{\blue{if} \> b \> \blue{then} \> \> 
                        p \> \blue{else} \> q}, 
                        \sigma \rangle \longrightarrow \pv{\prog{q}}{\sigma}
                }{
                        \langle \mathtt{b}, \sigma \rangle \longrightarrow^\star \mathtt{ff} 
                } 
                \hspace{2cm}
                \infer[(\text{wh$_2$})]{
                        \langle 
                        \mathtt{\blue{while} \> b \> \blue{do} \> \{ \> p \> \}}, \sigma \rangle
                        \longrightarrow^\star \sigma
                }{
                        \langle \mathtt{b}, \sigma \rangle \longrightarrow^\star \mathtt{ff}
                }
        \end{flalign*}
        %%
        \\[-30pt]
        %%
        \begin{flalign*}
                \infer[(\text{wh$_1$})]{
                        \langle \mathtt{\blue{while} \> b \> \blue{do} \> \{ \> p \> \}}, 
                        \sigma \rangle
                        \longrightarrow 
                        \pv{\prog{p} \> \blue{;} \> 
                        \mathtt{\blue{while} \> b \> \blue{do} \> \{ \> p \> \}}}{
                        \sigma}
                }{
                        \pv{\prog{b}}{\sigma} \longrightarrow^\star \prog{tt} 
                }
        \end{flalign*}
        \end{minipage}
        \caption{Small-step operational semantics for the while-language}
        \label{fig:small2}
        \end{figure}

        \begin{problem}
                Thus explain the meaning of the notation $\longrightarrow^\star$.
        \end{problem}

        \begin{problem}
                Calculate via the semantics in Figure~\ref{fig:small2} the
                chains of transitions that arise from the two programs below.
                \[
                        \prog{ x := 1} \> \blue{;} \> \prog{x := 2}
                        %
                        \hspace{3cm}
                        %
                        \prog{ \blue{if} } 
                        \> \> \prog{tt} \> \> \prog{ \blue{then}} \> \>
                        \prog{ x := 1} \> \blue{;} \> \prog{x := 2}
                        \> \> \prog{ \blue{else}}
                        \> \> \prog{ x := 3 }
                \]
                Finally give an example of a program that generates an
                \emph{infinite} sequence of transitions. Justify your choice
                mathematically :-).
        \end{problem}

\subsection{Big-step}

We now lean over our big-step operational semantics. Recall that it abstracts
away from all intermediate computational steps performed in the context of the
small-step semantics. It is therefore better suited  for calculating outputs
and for studying program equivalence (among other things).  The big-step
semantics of our while-language is defined in
Figure~\ref{fig:big}.

\begin{problem}
        Again in the lectures we did not see a big-step operational
        semantics for the arithmetic expressions. Can you define one such
        semantics now?
\end{problem}


        \begin{figure}[h]
        \begin{minipage}{1\textwidth}
        \begin{flalign*}
                \infer[(\text{asg})]{
                        \langle \mathtt{x := e}, \sigma \rangle \Downarrow 
                        \sigma[\mathtt{v} / \mathtt{x}]
                }{
                       \langle \mathtt{e}, \sigma \rangle \Downarrow \mathtt{v}
                } 
                \hspace{2cm}
                \infer[(\text{seq})]{
                        \langle \mathtt{p \> \blue{ ;} \> q}, \sigma \rangle \Downarrow \sigma''
                }{
                        \langle \mathtt{p}, \sigma \rangle \Downarrow \sigma' \qquad
                        \langle \mathtt{q}, \sigma' \rangle \Downarrow \sigma'' 
                }
        \end{flalign*}
        %%
        \\[-30pt]
        %%
        \begin{flalign*}
                \infer[(\text{if$_1$})]{
                        \langle \mathtt{\blue{ if} \> b \> \blue{ then} \> \> p \> \blue{ else} \> q}, 
                        \sigma \rangle \Downarrow \sigma'
                }{
                        \langle \mathtt{b}, \sigma \rangle \Downarrow \mathtt{tt} \qquad
                        \langle \mathtt{p}, \sigma \rangle \Downarrow \sigma'
                } 
                %
                \hspace{2cm} 
                %
                \infer[(\text{if$_2$})]{
                        \langle \mathtt{\blue{ if} \> b \> \blue{ then} \> \> p \> \blue{ else} \> q}, 
                        \sigma \rangle \Downarrow \sigma'
                }{
                        \langle \mathtt{b}, \sigma \rangle \Downarrow \mathtt{ff} \qquad
                        \langle \mathtt{q}, \sigma \rangle \Downarrow \sigma'
                } 
        \end{flalign*}
        %%
        \\[-30pt]
        %%
        \begin{flalign*}
                \infer[(\text{wh$_1$})]{
                        \langle \mathtt{\blue{ while} \> b \> \blue{ do} \> \{ \> p \> \}}, \sigma \rangle
                        \Downarrow \sigma''
                }{
                        \langle \mathtt{b}, \sigma \rangle \Downarrow \mathtt{tt} \qquad
                        \langle \mathtt{p}, \sigma \rangle \Downarrow \sigma' \qquad
                        \langle \mathtt{\blue{ while} \> b \> \blue{ do} \> \{ \> p \> \}}, \sigma'
                        \rangle \Downarrow \sigma'' 
                }
        \end{flalign*}
        %%
        \\[-30pt]
        %%
        \begin{flalign*}
                \infer[(\text{wh$_2$})]{
                        \langle \mathtt{\blue{ while} \> b \> \blue{ do} \> \{ \> p \> \}}, \sigma \rangle
                        \Downarrow \sigma
                }{
                        \langle \mathtt{b}, \sigma \rangle \Downarrow \mathtt{ff}
                }
        \end{flalign*}
        \end{minipage}
        \caption{A big-step operational semantics for the while-language}
        \label{fig:big}
        \end{figure}

\begin{problem}
Calculate via the semantics in Figure~\ref{fig:big} the outputs of the two
programs below.
                \[
                        \prog{ x := 1} \> \blue{;} \> \prog{x := 2}
                        %
                        \hspace{3cm}
                        %
                        \prog{ \blue{if} } 
                        \> \> \prog{tt} \> \> \prog{ \blue{then}} \> \>
                        \prog{ x := 1} \> \blue{;} \> \prog{x := 2}
                        \> \> \prog{ \blue{else}}
                        \> \> \prog{ x := 3 }
                \]
Reflect on the differences between calculating the outputs of these two
programs w.r.t. the small-step and big-step semantics.
\end{problem} 

Now that we have two different complementary semantics for our while-language
(small-step and big-step) it is useful to prove that they `agree' with each
other, or more formally that the equivalence below holds
       \[
                \pv{\prog{p}}{\sigma} \longrightarrow^\star \sigma'
                \text{ iff } \pv{\prog{p}}{\sigma} \Downarrow \sigma'
       \]
for every program $\prog{p}$ and states $\sigma,\sigma'$.  Here we will prove
just one side of the equivalence.
\begin{problem}
       Consider programs $\prog{p}$ and $\prog{p'}$ and states
       $\sigma,\sigma',\sigma''$. Prove by induction (over the depth of
       derivation trees) the following implication.
       \[
               \pv{\prog{p}}{\sigma} \longrightarrow \pv{\prog{p'}}{\sigma'}
               \Downarrow \sigma'' \text{ implies } \pv{\prog{p}}{\sigma} \Downarrow \sigma''
       \]
       Then use this last result to prove the implication below.
       \[
                \pv{\prog{p}}{\sigma} \longrightarrow^\star \sigma'
                \text{ implies } \pv{\prog{p}}{\sigma} \Downarrow \sigma'
       \]
       Are you able to prove the converse implication?
\end{problem}

Right, let us now recall the following (tentative) notion of program of
equivalence: two programs $\prog{p}$ and $\prog{q}$ will be equivalent (in
symbols $\prog{p} \equiv_o \prog{q}$) if the equivalence below holds.
       \[
                \pv{\prog{p}}{\sigma} \Downarrow \sigma'
                \text{ iff } \pv{\prog{q}}{\sigma} \Downarrow \sigma'
                \hspace{1cm} (\text{for all states } \sigma,\sigma')
       \]
\begin{problem}
        Prove the two equivalences propounded below.
        \begin{itemize}
                \item $\mathtt{(p \> \blue{ ;} \> q) \> {\blue ;} \> r} \equiv_o
                        \mathtt{p \> \blue{ ;} \> (q \> {\blue ;} \> r)}$
                \item $\mathtt{(\blue{ if} \> b \> \blue{ then} \> p \> \blue{ else} \> q) 
                      \> \blue{ ;} \> \prog{r}} \equiv_o 
                      \mathtt{\blue{ if} \> b \> \blue{ then} \> p \> \blue{ ;} \> r \> 
                      \blue{ else} \> q \> \blue{ ;} \> r}$ 
        \end{itemize}
\end{problem}

Recall now our second notion of program equivalence (the one that compilers
tacitly use). It is usually referred to as \emph{contextual equivalence} and
recurs to the following grammar of `contexts'.
\[              
                C ::= [-] \mid C \> \prog{ \blue{;}} \> \prog{p} 
                \mid 
                \prog{p} \> \prog{ \blue{;}} \> C 
                \mid 
                \mathtt{\blue{ if} \> b \> \blue{ then} \>} C \mathtt{\> \blue{ else} \> p} 
                \mid
                \mathtt{\blue{ if} \> b \> \blue{ then} \>} \prog{p} \mathtt{\> \blue{ else} \> C} 
                \mid
                \mathtt{\blue{ while} \> b \> \blue{ do} \> \{ \>} C \mathtt{\> \}}
\]
Two programs $\prog{p}$ and $\prog{q}$ will be contextually equivalent (in
symbols $\prog{p} \equiv \prog{q}$) if the equivalence below holds.
\[
               \pv{C[\prog{p}]}{\sigma} \Downarrow \sigma'
               \text{ iff } \pv{C[\prog{q}]}{\sigma} \Downarrow \sigma'
               \hspace{1cm} (\text{for all states } \sigma,\sigma' \text{ and contexts } C)
\]
\begin{problem}
        The natural question then arises of whether the equivalence below holds
        \[
                \prog{p} \equiv \prog{q} \text{ iff } \prog{p} \equiv_o \prog{q} 
        \]
        The left-to-right direction is clear, but what about the inverse direction?
        Can you prove it? Where do you get stuck?
\end{problem}

\section{Denotational Semantics}

Ok, we now focus on the denotational semantics of the previous lectures.
Recall that not only it helped us solve the previous problem, it also allowed
us to `import' knowledge from other mathematical theories (\eg\ `program
calculus'). 

A main challenge of this semantics was the interpretation of while-loops which
required us to touch on the surface of Domain theory and fixed-points. We start
from this end, in particular with the notion of a poset.
\begin{problem}
        Consider a set $\mathrm{State}$ of states. Show that it is a partially
        ordered set (poset) when it comes equipped with equality as the partial
        order.  Show as well that the pair $(\Pow (X),\subseteq)$ (for a given
        set $X$) is a partial order.

        Next, recall that given a poset $(X, \leq_X)$ we were able to define a
        new poset $(X_\bot, \leq)$. Thus prove that the latter is indeed a
        poset.  Finally given two sets $X$ and $Y$ prove that the set $[X,Y]$
        of maps between $X$ and $Y$ is a poset whenever $Y$ is a poset.
\end{problem}

Let us now go beyond posets and shift our attention to $\omega$-CPOs. We start
by revisiting the previous problem from the latter's perspective.

\begin{problem}
        Consider a set $\mathrm{State}$ of states. Show that it is an
        $\omega$-CPO when it comes equipped with equality as the partial order.
        Show as well that the pair $(\Pow (X),\subseteq)$ (for a given set $X$)
        is an $\omega$-CPO.

        Next, recall that given an $\omega$-CPO $(X, \leq_X)$ we were able to
        define a new $\omega$-CPO $(X_\bot, \leq)$. Thus prove that the latter
        is indeed an $\omega$-CPO.  Finally given two sets $X$ and $Y$ prove
        that the set $[X,Y]$ of maps between $X$ and $Y$ is a $\omega$-CPO
        whenever $Y$ is an $\omega$-CPO.
\end{problem}

Another key notion for defining the interpretation of while-loops was that of a
\emph{continuous} map $f : X \to Y$ (between $\omega$-CPOs $X$ and $Y$). Just
to recall, a map $f$ is called continuous if it satisfies the following
conditions.
\begin{flalign*}
        x_1 \leq x_2 & \text{ entails } f(x_1) \leq f(x_2) \qquad (\text{for all $x_1,x_2 \in X$ })
        \\
        f(\vee_n x_n) & = \vee_n f(x_n) \qquad 
        (\text{for every monotone sequence $(x_n)_{n \in \Nats}$}) 
\end{flalign*}

\begin{problem}
        Prove that the composition $g \comp f : X \to Z$ of continuous maps
        $f : X \to Y$ and $g : Y \to Z$ is continuous.
\end{problem}

These last three problems contain some of the necessary ingredients to
interpret while-loops as fixpoints. Another ingredient is Kleene's fixpoint
theorem which tells us how to build (the least) fixpoint of a continuous map $f
: X \to X$ on an $\omega$-CPO $X$ with a bottom element. The fixpoint (denoted
by $\mathrm{lfp}\, f$) is explicitly built as,
\[
        \mathrm{lfp}\, f = \vee_{n \in \Nats}\, f^n(\bot)
\]
\begin{problem}
        Show that $\mathrm{lfp}\, f$ is indeed the least fixpoint of $f$.
\end{problem}

Finally after some work we obtained the denotational semantics detailed in
Figure~\ref{fig:denot}.

\begin{figure}[h]
\begin{minipage}{1\textwidth}
\begin{flalign*}
        \sem{\prog{x : = e}} & = \sigma \mapsto \sigma[\sem{\prog{e}}/\prog{x}] \\[2pt]
        \sem{\prog{p} \> \blue{ ;} \> \prog{q}} & 
        = \sem{\prog{q}} \comp \sem{\prog{p}} \\[5pt]
        \sem{\mathtt{\blue{ if} \> b \> \blue{ then} \> p \> \blue{ else} \> q}}
                                                & 
                                                = [\sem{\prog{p}},\sem{\prog{q}}] \comp
                                                \mathrm{dist} \comp \pv{\sem{\prog{b}}}{\id}
                                                \\[5pt]
        \sem{\mathtt{\blue{ while} \> b \> \blue{ do } \> \{ \> p \> \}} }
                                                & \> {=} \> \mathrm{lfp}\ \Big (k \mapsto
        [k \comp 
        \sem{\prog{p}},\id] \comp
        \mathrm{dist} \comp \pv{\sem{\prog{b}}}{\id} \Big )
\end{flalign*}
\end{minipage}
\caption{A denotational semantics for the while-language.}
\label{fig:denot}
\end{figure}

We now have three different complementary semantics for our while-language
(small-step, big-step, and denotational). It is therefore useful to prove that they all
`agree' with each other. Technically we achieve this by proving the equivalence below 
       \[
              \pv{\prog{p}}{\sigma} \Downarrow \sigma' \text{ iff }
              \sem{\prog{p}}(\sigma) = \sigma' 
      \]
for every program $\prog{p}$ and states $\sigma,\sigma'$.  Here we will prove
just one side of the equivalence.
\begin{problem}
       Consider programs $\prog{p}$ and $\prog{p'}$ and states
       $\sigma,\sigma'$. Prove by induction (over the depth of
       derivation trees) the following implication.
       \[
               \pv{\prog{p}}{\sigma}
               \Downarrow \sigma' \text{ implies } \sem{\prog{p}}(\sigma) = \sigma' 
       \]
       (Hint: Use the fixpoint equation).
\end{problem}

From the previous equivalence between the denotational and big-step semantics
we can easily establish that for all programs $\prog{p}$ and $\prog{q}$ the
following equivalence holds.
\[
        \prog{p} \equiv_o \prog{q} \text{ iff } \sem{\prog{p}} = \sem{\prog{q}}
\]
But we also wish to learn more about the notion of contextual equivalence
($\equiv$), in particular we would like to know whether  $\prog{p} \equiv_o
\prog{q}$ entails $\prog{p} \equiv \prog{q}$. Can denotational semantics help
us in this quest?

\begin{problem}
        Consider two programs $\prog{p}$ and $\prog{q}$. Prove the following
        implication.  
        \[
                \sem{\prog{p}} = \sem{\prog{q}} \text{ entails }
                \Big ( \text{for all contexts } C. \,
                \sem{C[\prog{p}]} = \sem{C[\prog{q}]} \Big )
        \]
\end{problem}

Great, we established that contextual equivalence reduces to \emph{equality} of
denotations! We can thus use our denotational semantics (and all mathematical
theories it inherits, such as `program calculus') to prove equivalences used by
programmers and compilers all the time.

\begin{problem}
        Prove via the denotational semantics that the following equivalences
        indeed hold.
        \begin{itemize}
                \item $\mathtt{(p \> \blue{ ;} \> q) \> \blue{ ;} \> r} \equiv
                        \mathtt{p \> \blue{ ;} \> (q \> \blue{ ;} \> r)}$
                \item $\mathtt{(\blue{if} \> b \> \blue{then} \> p \> \blue{else} \> q) 
                      \> \blue{ ;} \> \prog{r}} \equiv 
                      \mathtt{\blue{ if} \> b \> \blue{then} \> p \> \blue{ ;} \> r \> 
                      \blue {else} \> q \> \blue {;} \> r}$ 
              \item $\mathtt{\blue {while} \> b \> \{ p \} } \equiv
                        \mathtt{\blue{if} \> b \> \blue{then} \> p \> {\blue ;} \>
                        \mathtt{\blue{ while} \> b \> \{ p \} } 
                        \> \blue{ else} \> skip}$
                \item $\mathtt{\blue{ while} \> b \> \{ p \} } \> \blue{ ;} \> \mathtt{q} \equiv
                      \mathtt{\blue{if} \> b \> \blue{ then} \> p \> \blue{ ;}  \> \blue{ while} \> b \> \{ p \}
                      \> \blue{ ;} \> q
                      \> \blue{ else} \> q}$
                \item $\mathtt{\blue{ while} \> ff \> \{ p \} \> \blue{ ;} \> q } \equiv \mathtt{q}$
                \item $\mathtt{\blue{ while} \> tt \> \{ p \} } \equiv
                      \mathtt{\blue{ while} \> tt \> \{ q \} }$
        \end{itemize}
        Reflect on the differences between performing this exercise with the
        denotational, small-step, or big-step semantics.
\end{problem}

\begin{problem}
        Consider the program $\mathtt{\blue{ while} \> tt \> \{ p \} }$ and a
        state $\sigma$. Can you prove that there is no state $\sigma'$ such
        that $\mathtt{\blue{ while} \> tt \> \{ p \} } \Downarrow \sigma'$?
        This can be a bit hard to prove, because it is usually cumbersome to
        prove \emph{non-existence} of things. Perhaps the denotational
        semantics can help us in this quest. What do you think?
\end{problem}

\section{Axiomatic Semantics}

Finally we lean over our axiomatic semantics, which as you know has a more
logical nature and it is therefore better suited to study \eg\ program
correctness.  This style of semantics puts the notion of proposition (\ie\
condition) at center stage, which of course raises the question of which logic
to choose for writing and reasoning about such propositions.  We started by
observing that one does not actually need to choose a logic right from the
outset, but rather one makes basic assumptions about whatever logic is chosen.
One of these assumptions is that every proposition $\Phi$ will correspond to a
subset $\sem{\Phi} \subseteq \mathrm{State}_\bot$. From this we worked out what
a Hoare triple should mean mathematically, \emph{viz.}
\[
         \{ \Phi \} \, \prog{p} \, \{ \Psi \} \qquad \text{ means } \qquad 
         \Big (\forall x \in \mathrm{State}_\bot. \> x \in \sem{\Phi} 
         \text{ entails } \sem{\prog{p}}(x) \in \sem{\Psi} \, \Big )
\]

\begin{problem}
        Argue informally whether the following Hoare triples hold.
        \begin{itemize}
                \item $\{ \prog{tt} \} \> \> \prog{\blue{ while} \> \> tt \> \> skip} \> \> \{ \prog{ff} \}$
                \item $\{ \prog{tt} \} \> \> \prog{\blue{ if} \>  b \> \blue{ then} \> \> 
                        x:=2 \> \> \blue{ else} \> \> x:= 3} \> \> \{ \prog{x \geq 2} \}$
                \item $\{ \prog{x = a \wedge y = b} \} \> \>
                        \prog{x:= y \> \blue{ ;} \> y := x} \> \> \{ \prog{ x = b \wedge y = a }\}$
                \item $\{ \prog{x = a \wedge y = b} \} \> \>
                        \prog{aux := x \> \blue{ ;} \> 
                        x:= y \> \blue{ ;} \> y := aux} \> \> \{ \prog{ x = b \wedge y = a }\}$
        \end{itemize}
\end{problem}

We then continued exploring the mathematical meaning of Hoare triples, and
among other things we made the following observation.
\[
         \Big (\forall x \in \mathrm{State}_\bot. \> x \in \sem{\Phi} 
         \text{ entails } \sem{\prog{p}}(x) \in \sem{\Psi} \, \Big )
         \> \text{ iff } \>
         \sem{\Phi} \subseteq \sem{\prog{p}}^{-1}(\sem{\Psi})
\]
This suggests not only we study the validity of Hoare triples from the point of
view of inverse images, but also that it is informative to calculate
$\sem{\prog{p}}^{-1}(\sem{\Psi})$ (\ie\ the intrepretation of the weakest
pre-condition w.r.t. the post-condition $\Psi$). Of course it would be a hassle
to calculate pre-images of some set directly. The weakest pre-condition
semantics we saw circumvents this problem, among other things. It is detailed
in Figure~\ref{fig:ax1}.

\begin{figure}[h]
\begin{minipage}{1\textwidth}
\begin{flalign*}
                \mathrm{wp}\,(\prog{x : = e}, \Phi) & = \Phi[\prog{e}/\prog{x}]
                \\[5pt]
                %
                \mathrm{wp}\,(\prog{p} \> \blue{ ;} \> \prog{q}, \Phi) & 
                = \mathrm{wp} \, (\prog{p}, \mathrm{wp} \, (\prog{q}, \Phi))  
                \\[5pt]
                %
                \mathrm{wp}\,({\mathtt{\blue{ if} \> b \> \blue{ then} \> p \> \blue{ else} \> q}}, \Phi)
                & 
                =
                \prog{b} \wedge \mathrm{wp}\, (\prog{p},\Phi) \, \vee \,
                \prog{\neg b} \wedge \mathrm{wp} \, (\prog{q}, \Phi)
                \\[5pt]
                \mathrm{wp} \, ({\mathtt{\blue{ while} \> b \> \blue{ do} \> \{ \> p \> \}} }, \Phi)
                & = \bigwedge_{n \in \Nats} \Psi_n
                \\[10pt]
        \Psi_0 & = \prog{tt} \\
        \Psi_{n+1} & = \neg \prog{b} \wedge \Phi \, \vee \, \prog{b} \wedge 
        \mathrm{wp} \, (\prog{p}, \Psi_n)
\end{flalign*}
\end{minipage}
\caption{A weakest pre-condition semantics for the while-language.}
\label{fig:ax1}
\end{figure}

\begin{problem}
 Calculate the weakest pre-conditions w.r.t. the following pairs.
 \begin{itemize}
         \item $(\prog{x := y}, \>  \prog{x} \geq 1)$
         \item $(\prog{\blue{ if} \>  b \> \blue{ then} \> \> 
                 x:=2 \> \> \blue{ else} \> \> x:= 3}, \> \prog{x \geq 2})$
         \item $(\prog{x:= y \> \blue{ ;} \> y := x}, \> \prog{ x = b \wedge y = a })$
         \item $(\prog{ aux := x \blue{ ;} \> 
                 x:= y \> \blue{ ;} \> y := aux}, \> \prog{ x = b \wedge y = a })$
 \end{itemize}
\end{problem}

\begin{problem}
  Prove that $\mathrm{wp} \, (\prog{p}, \prog{tt}) = \prog{tt}$ for every
  program $\prog{p}$.
\end{problem}

Following our traditions, we should connect this new semantics to the previous
ones. And indeed we saw in the lectures that the following equation holds for
every program $\prog{p}$ and condition $\Phi$.
\[
        \sem{\mathrm{wp} \, (\prog{p}, \Phi)} = \sem{\prog{p}}^{-1}(\sem{\Phi})
\]
which connects us to denotational semantics.  As mentioned in the lectures, the
respective proof can be neatly established via Domain theory. Unfortunately
the techniques involved are a bit more advanced than the ones addressed in the
lectures :-(.

\begin{problem}
    Resort to the denotational semantics to prove the following equivalences.
    \begin{itemize}
                \item $\mathrm{wp} \, (\prog{p}, \prog{tt}) \equiv \prog{tt}$
                \item $\mathrm{wp} \, (\prog{p}, \Phi \wedge \Psi )
                \equiv \mathrm{wp} \, (\prog{p}, \Phi) \wedge 
                \mathrm{wp} \, (\prog{p}, \Psi )$
   \end{itemize}
   Reflect on the differences between showing the first equivalence in this problem and your
   solution to the previous problem.
\end{problem}

Next, we saw how to generate a calculus from this pre-condition semantics for
showing the validity of Hoare triples. This calculus is prominently simple: in
fact it only contains one rule, \emph{viz.}
\[
                \infer{\vdash \{ \Phi \} \, \prog{p} \, \{ \Psi \}}
                { \vdash \Phi \rightarrow \mathrm{wp} \, (\prog{p}, \Psi) }
\]

\begin{problem}
        Prove that this calculus is sound (\ie\ correct) and relatively complete.
\end{problem}

Later on we observed that although the calculus is interesting it is a bit `too
rigid' to prove the validity of Hoare triples: indeed the fact that it is
strictly based on weakest pre-conditions and infinite conjunction makes less
amenable to practical uses. This brings us back to old friends, namely the
Hoare calculus that you saw in `algorithms and complexity' and which is detailed
in Figure~\ref{fig:Hoare}.

\begin{figure}[h]
\begin{minipage}{1\textwidth}
\begin{flalign*}
                \infer{ \vdash_H \{ \Phi \} \> \prog{skip} \> \{ \Phi \} }{}
                \hspace{2.5cm}
                \infer{ \vdash_H \{ \Phi[\prog{e}/\prog{x}] \} \> \prog{ x : = e } \>
                \{ \Phi \} }{}
\end{flalign*}
\begin{flalign*}
                \infer { \vdash_H \{ \Phi \} \> \prog{p \> \blue{ ;} \> q} \> \{ \Xi \} }{
                        \vdash_H \{ \Phi \} \> \prog{p} \> \{ \Psi \}
                        \qquad
                        \vdash_H \{ \Psi \} \> \prog{q} \> \{ \Xi \}
                }
                \hspace{2.5cm}
                \infer{
                        \vdash_H \{ \Phi \} \> \prog{\blue{ while}} \> \prog{b} \>
                        \{ \> \prog{p} \> \} \> 
                        \{ \neg \prog{b} \wedge \Phi \}
                }{
                        \vdash_H
                        \{ \Phi \wedge \prog{b} \} \>  \prog{p}  \> \{ \Phi \}
                }
\end{flalign*}
\begin{flalign*}
                \infer {
                        \vdash_H \{ \Phi \} \> \prog{\blue{ if} \> b}
                        \> \prog{\blue{ then}} \>
                        \prog{p} \> \prog{\blue{ else}} \> \prog{q}
                        \> \{ \Psi \}
                }{
                        \vdash_H \{ \prog{b} \wedge \Phi \} \> \prog{p} \> \{ \Psi \}
                        \qquad
                        \vdash_H \{ \prog{\neg b} \wedge \Phi \} \> \prog{q} \> \{ \Psi \}
                }
\end{flalign*}
\begin{flalign*}
                \infer{ \vdash_H \{ \Phi \} \> \prog{p} \> \{ \Psi \} }{
                        \vdash \Phi \rightarrow \Psi
                        \quad
                        \vdash_H \{ \Psi \} \> \prog{p} \> \{ \Xi \}
                        \quad
                        \vdash \Xi \rightarrow \Omega
                }
\end{flalign*}
\end{minipage}
\caption{Hoare calculus for our while-language.}
\label{fig:Hoare}
\end{figure}

\begin{problem}
        Prove by induction (on the depth of derivation trees) that the Hoare
        calculus is sound (correct), \ie\ that the following implication holds
        for every program $\prog{p}$ and conditions $\Phi, \Psi$.
        \[
                \vdash_H \{ \Phi \} \, \prog{p} \, \{ \Psi \}
                \text{ entails }
                \sem{\Phi} \subseteq \sem{\prog{p}}^{-1}(\sem{\Psi})
        \]
        You can skip the case of while-loops for it is much harder than the other ones..
\end{problem}
\end{document}
