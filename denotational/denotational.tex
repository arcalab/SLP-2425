\documentclass{beamer}
\usepackage{etex} % fixes new-dimension error
\usepackage{lmodern}
\usepackage[T1]{fontenc}
\usetheme{metropolis}
\metroset{block=fill}
%\usetheme{Boadilla}

\setbeamertemplate{footline}
{
  \leavevmode%
  \hbox{%
  \begin{beamercolorbox}[wd=.4\paperwidth,ht=2.25ex,dp=1ex,center]{author in head/foot}%
    \usebeamerfont{author in head/foot}\insertshortauthor
  \end{beamercolorbox}%
  \begin{beamercolorbox}[wd=.5\paperwidth,ht=2.25ex,dp=1ex,center]{title in head/foot}%
    \usebeamerfont{title in head/foot}\insertsection
  \end{beamercolorbox}%
  \begin{beamercolorbox}[wd=.1\paperwidth,ht=2.25ex,dp=1ex,right]{date in head/foot}%
    \insertframenumber{} / \inserttotalframenumber\hspace*{2ex} 
  \end{beamercolorbox}}%
  \vskip0pt%
}
\makeatother
\setbeamertemplate{navigation symbols}{}
%----------------------------------------------------------------------------
\usepackage{graphicx,amsmath}
\usepackage{stmaryrd} % cf. interleave
\usepackage{booktabs}
\usepackage{amscd}
\usepackage{multicol}
\usepackage[absolute,overlay]{textpos}
\usepackage{alltt}
\usepackage{proof}
\usepackage{listings}
\usepackage[all]{xy}
\input{macros}
%------ using pstricks (rnode etc) ------------------------------------------
\usepackage{pstricks,pst-node,pst-text,pst-3d}

% ------ using color ---------------------------------------------------------
\newrgbcolor{goldenrod}{.80392 .60784 .11373}
\newrgbcolor{darkgoldenrod}{.5451 .39608 .03137}
\newrgbcolor{brown}{.15 .15 .15}
\newrgbcolor{darkolivegreen}{.33333 .41961 .18431}
\def\gold#1{{\goldenrod #1}}
\def\dgold#1{{\alert{#1}}}
\def\dkb#1{{\blue #1}}
\def\tdkb#1{\textbf{\darkblue #1}}
\def\gre#1{{\darkolivegreen #1}}
\def\gry#1{{\gray #1}}
\def\rdb#1{{\red #1}}
%----------------------------------------------------------------------------

\AtBeginSection[]
{
    \begin{frame}
        \frametitle{Table of Contents}
        \tableofcontents[currentsection]
    \end{frame}
}

\author[Renato Neves]{Renato Neves}

% logos of institutions
\titlegraphic{
  \begin{textblock*}{5cm}(6.5cm,7.4cm)
     \includegraphics[scale=0.06]{images/uminho.png}\hspace*{.85cm}~%
  \end{textblock*}
  \begin{textblock*}{5cm}(9.2cm,7.45cm)
    \includegraphics[scale=0.50]{images/haslab.pdf}
  \end{textblock*}
}

% No date
\date{}

\begin{document}

\title{Denotational Semantics}

\frame[plain]{\titlepage}


\begin{frame}{Semantics for Every Season}

        \hspace*{+5pt}\makebox[\linewidth][c]{%
        \begin{tabular}{ l l }
        Operational semantics & How a program operates
        \\
        \alert{\underline{Denotational semantics}} & What a program is 
        \\
        Axiomatic semantics & Which logical properties a program satisfies
        \end{tabular}
        }

\end{frame}

\section{Motivation}

\begin{frame}{Compiler Correctness and Contextual Equivalence}
        Adopted a natural notion of \alert{\underline{equivalence}}         
        \begin{flalign*}
                \text{ $\mathtt{p} \equiv_o \mathtt{q}$ iff $\Big ($for
                every $\sigma$. }
                \langle \mathtt{p}, \sigma \rangle \Downarrow \mathtt{\sigma'} 
                \text{ iff } \langle \mathtt{q}, \sigma \rangle \Downarrow \mathtt{\sigma'}
                \Big )
        \end{flalign*}

        \pause
        \medskip
        Compilers adopt a \alert{\underline{stronger}} version
        \begin{flalign*}
                \text{ $\mathtt{p} \equiv \mathtt{q}$ iff }
                \Big ( \text{for
                every context $C$. }
                C[\mathtt{p}] \, \equiv_o C[\mathtt{q}]
                \Big )
        \end{flalign*}

        \pause
        \bigskip
        \centering
        \fbox{ Why ? }
\end{frame}

\begin{frame}{Contextual Equivalence}

        \begin{block}{Contexts}
                $C ::= [-] \mid C \wedge \prog{b} \mid \prog{b} \wedge C \mid \neg C$
        \end{block} 

        \bigskip
        \begin{block}{Exercise}
                Prove the equivalence $\prog{b_1} \equiv_o \prog{b_2} 
                \text{ iff } \prog{b_1} \equiv \prog{b_2}$
        \end{block}

        \pause
        \bigskip
        \begin{block}{Exercise}
        Repeat the previous exercise now for arithmetic expressions
        \end{block}
\end{frame}

\begin{frame}{Contextual Equivalence}

        \begin{block}{Contexts}
                $C ::= [-] \mid C \> \prog{\blue ;} \> \prog{p} 
                \mid 
                \mathtt{{\blue if} \> b \> {\blue then} \>} C \mathtt{\> {\blue else} \> p} 
                \mid
                \mathtt{{\blue while} \> b \> {\blue do} \> \{ \>} C \mathtt{\> \}}
                \mid \dots$
        \end{block} 

        \pause
        \bigskip
        \bigskip
        \centering
        \fbox{Can one still prove $\prog{p} \equiv_o \prog{q} \text{ iff }
        \prog{p} \equiv \prog{q}$ ?}

\end{frame}

\begin{frame}{Next Challenge: Programs as part of a Mathematical Theory}
        
        \begin{center}
        \fbox{Programming language} \hspace{0.5cm} \scalebox{1.5}{${\hookrightarrow}$} 
        \hspace{0.5cm} \fbox{Mathematical theory}
        \end{center}

        \medskip
        The latter include \emph{e.g.}
        \begin{itemize}
                \item functions (recall program calculus)
                \item linear algebra
                \item relations
                \item \alert{\underline{domain theory}} (theory of computability and beyond)
                \item \dots
        \end{itemize}
\end{frame}

\section{My first denotational semantics}

\begin{frame}{Boolean Terms and their Denotational Semantics}

        \begin{block}{\vspace*{-3.5ex}}
        \[
                \prog{b} ::= \prog{x} \mid \prog{b} \wedge \prog{b} \mid \neg \prog{b}
        \]
        \end{block} 

        \bigskip
        Terms interpreted as \alert{\underline{functions}}
        $\sem{\prog{b}} : State \to 2$

        Term operations interpreted via the \alert{\underline{boolean algebra}} $2$ 
        \bigskip
        \begin{align*}
                \sem{\prog{x}}(\sigma) & = \sigma (\prog{x}) \\
                \sem{\prog{b_1} \wedge \prog{b_2}} & 
                = (\wedge) \comp \pv{\sem{\prog{b_1}}}{\sem{\prog{b_2}}} \\
                \sem{\neg \prog{b}} & = (\neg) \comp \sem{\prog{b}}
        \end{align*}
\end{frame}

\begin{frame}{Big-step meets Denotational \dots }

        \begin{theorem}
                $\pv{\prog{b}}{\sigma} \Downarrow v \text{ iff }
                \sem{\prog{b}}(\sigma) = v$
        \end{theorem}

        \begin{proof}
                Straightforward \alert{\underline{induction}}
        \end{proof}

        \bigskip
        \bigskip
        \begin{corollary}
                $\prog{b_1} \equiv \prog{b_2} \> \text{ iff } \>
                \prog{b_1} \equiv_o \prog{b_2} \> \text{ iff } \>
                \sem{\prog{b_1}} = \sem{\prog{b_2}}$
        \end{corollary}
\end{frame}

\begin{frame}{Profits !}
        
        We can now reduce checking equivalence to your favorites \dots
        \begin{center}
                \fbox{
                        \alert{\underline{Program calculus}} and \alert{\underline{Boolean algebra}}
                }
        \end{center}
        
        \pause
        \begin{example}
        \small{
        \vspace{-0.2cm}
        \begin{align*}
                \sem{\prog{b_1} \wedge \prog{b_2}} & =
                (\wedge) \comp \pv{\sem{\prog{b_1}}}{\sem{\prog{b_2}}}
                \\
                                                   & = (\wedge) \comp \mathrm{sw}
                                                   \comp 
                                                   \pv{\sem{\prog{b_1}}}{\sem{\prog{b_2}}}
                                                   \\
                                                   & = (\wedge) \comp \pv{\pi_2}{\pi_1}
                                                   \comp 
                                                   \pv{\sem{\prog{b_1}}}{\sem{\prog{b_2}}}
                                                   \\
                                                   & = (\wedge) \comp \pv{\pi_2
                                                   \comp \pv{\sem{\prog{b_1}}}{
                                                   \sem{\prog{b_2}}}}{\pi_1
                                                   \comp \pv{\sem{\prog{b_1}}}{\sem{\prog{b_2}}}}
                                                   \\
                                                   & = (\wedge) \comp 
                                                   \pv{\sem{\prog{b_2}}}{\sem{\prog{b_1}}}
                                                   \\
                                                   & = \sem{\prog{b_2} \wedge \prog{b_1}}
         \end{align*}
        }
        \end{example}
\end{frame}


\begin{frame}{Exercises}
                Show $\prog{b} \wedge \prog{b} \equiv \prog{b}$ (via the
                denotational semantics)
 
                Define a denotational semantics for arithmetic expressions 

                Show $\prog{e_1} + \prog{e_2} \equiv \prog{e_2} + \prog{e_1}$
                (via your denotational semantics)

                Prove the equivalence $\pv{\prog{e}}{\sigma} \Downarrow v
                \text{ iff } \sem{\prog{e}}(\sigma) = v$
\end{frame}

\section{Denotational semantics for a while-language}

\begin{frame}{Key Takeaways}
        Programs interpreted as \alert{\underline{functions}}
        $\sem{\prog{p}} : State_\bot \to State_\bot$

        $State_\bot = State \cup \{ \bot \}$ where $\bot$ represents
        \alert{\underline{non-termination}}

        Sequential composition is \alert{\underline{function composition}}
\end{frame}

\begin{frame}{Programs and their Denotational Semantics}

        \begin{block}{\vspace*{-3.5ex}}
        \begin{center}
        $\prog{p} ::= \mathtt{x} := \mathtt{e} \mid
	\mathtt{p \> {\blue ;} \> p} \mid
	\mathtt{{\blue if} \> b \> {\blue then} \> p \> {\blue else} \> p} \mid
	\mathtt{{\blue while} \> b \> {\blue do} \> \{ \> p \> \}}$
        \end{center}
	\end{block}
        \vspace{-0.5cm}
        \begin{align*}
                \sem{\prog{x : = e}} & = \sigma \mapsto \sigma[\sem{\prog{e}}/\prog{x}] \\
                \sem{\prog{p} \> {\blue ;} \> \prog{q}} & 
                = \sem{\prog{q}} \comp \sem{\prog{p}} \\
                \sem{\mathtt{{\blue if} \> b \> {\blue then} \> p \> {\blue else} \> q}}
                                                        & 
                                                        = [\sem{\prog{p}},\sem{\prog{q}}] \comp
                                                        \mathrm{dist} \comp \pv{\sem{\prog{b}}}{\id}
                \\
                \sem{\mathtt{{\blue while} \> b \> {\blue do} \> \{ \> p \> \}} }
                                                        & =\  \dots
        \end{align*}
\end{frame}

\begin{frame}{Big-step and Denotational Semantics}

        \alert{\underline{Danger, Will Robinson}}: no while-loops yet \dots

        \medskip
        \begin{theorem}
                $\pv{\prog{p}}{\sigma} \Downarrow \sigma' \text{ iff }
                \sem{\prog{p}}(\sigma) = \sigma'$
        \end{theorem}

        \begin{proof}
                Straightforward \alert{\underline{induction}}
        \end{proof}

        \bigskip
        \begin{corollary}
                $\prog{p} \equiv \prog{q} \> \text{ iff } \>
                \prog{p} \equiv_o \prog{q} \> \text{ iff } \>
                \sem{\prog{p}} = \sem{\prog{q}}$
        \end{corollary}
\end{frame}

\begin{frame}{Profits !}

        Recall when we had to prove
        \begin{itemize}
                \item $\mathtt{(p \> {\blue ;} \> q) \> {\blue ;} \> r} \equiv
                        \mathtt{p \> {\blue ;} \> (q \> {\blue ;} \> r)}$
                \item $\mathtt{({\blue if} \> b \> {\blue then} \> p \> {\blue else} \> q) 
                      \> {\blue ;} \> \prog{r}} \equiv 
                      \mathtt{{\blue if} \> b \> {\blue then} \> p \> {\blue ;} \> r \> 
                      {\blue else} \> q \> {\blue ;} \> r}$ 
        \end{itemize}
        with the \alert{\underline{big-step semantics}}

        \bigskip
        Show the same via the \alert{\underline{denotational semantics}}
\end{frame}

\begin{frame}{Programs and a Tentative Denotational Semantics}
        \begin{block}{\vspace*{-3.5ex}}
        \begin{center}
        $\prog{p} ::= \mathtt{x} := \mathtt{e} \mid
	\mathtt{p \> {\blue ;} \> p} \mid
	\mathtt{{\blue if} \> b \> {\blue then} \> p \> {\blue else} \> p} \mid
	\mathtt{{\blue while} \> b \> {\blue do} \> \{ \> p \> \}}$
        \end{center}
	\end{block}
        \vspace{-0.5cm}
        \begin{align*}
                \sem{\prog{x : = e}} & = \sigma \mapsto \sigma[\sem{\prog{e}}/\prog{x}] \\
                \sem{\prog{p} \> {\blue ;} \> \prog{q}} & 
                = \sem{\prog{q}} \comp \sem{\prog{p}} \\
                \sem{\mathtt{{\blue if} \> b \> {\blue then} \> p \> {\blue else} \> q}}
                                                        & 
                                                        = [\sem{\prog{p}},\sem{\prog{q}}] \comp
                                                        \mathrm{dist} \comp \pv{\sem{\prog{b}}}{\id}
                \\
                \sem{\mathtt{{\blue while} \> b \> {\blue do} \> \{ \> p \> \}} }
                                                        & \> {=} \>
                [\sem{\mathtt{{\blue while} \> b \> {\blue do} \> \{ \> p \> \}} } \comp 
                \sem{\prog{p}},\id] \comp
                \mathrm{dist} \comp \pv{\sem{\prog{b}}}{\id}
        \end{align*}

        \pause
        \smallskip
        \begin{flushright}
        \emph{\textbf{I'm very clear, Brexit does mean brexit}}

        \scriptsize{(Theresa May) \url{https://www.youtube.com/watch?v=oRDfFJAu6Bo}}
        \end{flushright}
\end{frame}

\section{Domain theory}

\begin{frame}{Partially Ordered Set}

        \begin{definition}[Poset]
                A set with a reflexive, anti-symmetric, and transitive
                relation $\leq$
        \end{definition}

        \begin{examples}
                \begin{itemize}
                        \item $(\Nats, \text{ the usual order $\leq$ on natural numbers})$
                        \item $(\Reals, \text{ the usual order $\leq$ on real numbers})$
                        \item $(X, \> \> =)$ \small{(for any set $X$)}
                        \item $(\Pow X, \subseteq)$ \small{(for any set $X$)}
                \end{itemize}
        \end{examples}

        \pause
        \smallskip
        In our context $x \leq y$ reads as
        \begin{center}
                \fbox{
                        $x$ \alert{\underline{less informative}} than $y$
                }
        \end{center}
\end{frame}

\begin{frame}{New Posets from Old Ones}

        \begin{block}{Addition of a bottom element}
                If $(X, \leq_X)$ is a poset then $(X_\bot, \leq)$ is
                a poset when defined as
                \begin{itemize}
                        \item $x_1 \leq x_2$  iff $x_1 \leq_X x_2$
                        \item $\bot \leq x$ ($\text{for all } x \in X$)
                \end{itemize}
        \end{block}

        $\bot$ is the least informative element, akin to non-termination

        \pause
        \medskip
        \begin{example}
                In what way is $State_\bot$ a poset ?
        \end{example}
\end{frame}

\begin{frame}{Data Aggregation}
        We wish to collect a \alert{\underline{chain of information}}
        \[
                x_1 \leq x_2 \leq x_3 \leq \dots
        \]
        into a \alert{\underline{single}} datum, denoted by $\vee_{i \in \Nats}\, x_i$

        \pause
        \bigskip
        This element should be \alert{\underline{more informative}}
        than any $x_j$ ($j \in \Nats$) \ie\
        \[
                x_j \leq \vee_{i \in \Nats}\, x_i
        \]
        \dots\ and contain \alert{\underline{no more information}} than the chain
        \ie\
        \[
                (\forall j \in \Nats. \, x_j \leq y) \Longrightarrow
                \vee_{i \in \Nats}\, x_i \leq y
        \]
\end{frame}

\begin{frame}{Posets + Data Aggregation}

        \begin{definition}[$\omega$-CPO]
                A poset with data aggregation as previously described
        \end{definition}

        \begin{examples}
                \begin{itemize}
                        \item $\Nats$ is \underline{not} an $\omega$-CPO but
                                $\Nats \cup \{ \infty \}$ is
                        \item $\Reals$ is \underline{not} an $\omega$-CPO but
                                $\Reals \cup \{ \infty \}$ and $[0,1]$ are
                        \item $(\Pow X, \subseteq)$ is an $\omega$-CPO for any set $X$
                \end{itemize}
        \end{examples}

        \pause
        \bigskip
        \begin{block}{Exercise}
                Show that $State_\bot$ is an $\omega$-CPO
        \end{block}
\end{frame}

\begin{frame}{Maps between $\omega$-CPOs}
        We wish that maps represent some form of
        \alert{\underline{computability}} \dots

        \dots\ and thus any old map will not do

        \pause
        \bigskip
        We will therefore enforce the following laws
        \begin{align*}
                f(\vee_n\, x_n) & = \vee_{n}\, f(x_n) & \text{(continuity)}
                \\
                x_1 \leq x_2 & \Rightarrow f(x_1) \leq f(x_2) & \text{(monotonicity)}
        \end{align*}
\end{frame}

\begin{frame}{Continuity $\simeq$ Computability ?}

        \begin{center}
        \fbox{
                What does it mean for $p : X \to \{ \bot \leq \top \}$ to be continuous ?
        }
        \end{center}

        \medskip
        Let $x \in X$ be given by a chain of \alert{\underline{finite}} approximations
        \[
                x_1 \leq x_2 \leq x_3 \, \dots
        \]
        \dots\ then deduce that
        \begin{align*}
                p(\vee_{n}\, x_n) = \top & \Longleftrightarrow \vee_{n}\, p(x_n) = \top
                \\
                                                   & \Longleftrightarrow \exists n.\, p(x_n) = \top
        \end{align*}
        \pause
        \ie\ $p$ \alert{\underline{terminates}} with $\top$ (true) for
        $x$ iff $p$ can evaluate a \alert{\underline{finite}} approximation of
        $x$ to $\top$
\end{frame}

\begin{frame}{Continuity $\simeq$ Computability ?}
        
        \begin{block}{Exercise}
                Show that $(\Pow \Nats, \subseteq)$ is an $\omega$-CPO
        \end{block}

        \begin{block}{Exercise}
                Is $\mathtt{isInfinite} : \mathcal{P}(\Nats) \to \{
                \bot \leq \top \}$ continuous ?
        \end{block}
\end{frame}

\begin{frame}{Continuity $\simeq$ Computability ?} 

        \begin{align*}
                \sem{\prog{x : = e}} & = \sigma \mapsto \sigma[\sem{\prog{e}}/\prog{x}] \\
                \sem{\prog{p} \> {\blue ;} \> \prog{q}} & 
                = \sem{\prog{q}} \comp \sem{\prog{p}} \\
                \sem{\mathtt{{\blue if} \> b \> {\blue then} \> p \> {\blue else} \> q}}
                                                        & 
                                                        = [\sem{\prog{p}},\sem{\prog{q}}] \comp
                                                        \mathrm{dist} \comp \pv{\sem{\prog{b}}}{\id}
                \\
                \sem{\mathtt{{\blue while} \> b \> {\blue do} \> \{ \> p \> \}} }
                                                        & =\  \dots \dots
        \end{align*}

        \vfill
        \begin{center}
        \fbox{ Are all maps $\sem{\prog{p}}$ continuous ? }
        \end{center}
        \pause
        \begin{center}
        Yes, just use \alert{\underline{program calculus}} `with
        \alert{\underline{continuous}} maps'
        \end{center}
\end{frame}

\begin{frame}{Fixpoints for While-loops}

        \begin{definition}
                $x \in X$ is a \alert{\underline{fixpoint}} of $f :
                X \to X$ if $f(x) = x$
        \end{definition}

        A notion with powerful applications in diverse fields
        \begin{itemize}
                \item economics (game theory)
                \item dynamical systems (equilibrium points)
                \item automata theory
                \item essentially everywhere \dots
        \end{itemize}

        \pause
        \bigskip
        While-loops will be fixpoints in our semantics
\end{frame}

\begin{frame}{Fixpoints for While-loops}

        \begin{center}
                \fbox{
                       \dots\ a fixpoint of which function ? 
                }
        \end{center}

        \pause
        \bigskip
        Recall our previous approach
        \[
                \sem{\mathtt{{\blue while} \> b \> {\blue do} \> \{ \> p \> \}} }
                                                         \> {=} \>
                [\sem{\mathtt{{\blue while} \> b \> {\blue do} \> \{ \> p \> \}} } \comp 
                \sem{\prog{p}},\id] \comp
                \mathrm{dist} \comp \pv{\sem{\prog{b}}}{\id}
        \]
        It states that $\sem{\mathtt{{\blue while} \> b \> {\blue do} \> \{ \>
        p \> \}} }$ is a fixpoint of 
        \[
               k \longmapsto 
               [k \comp 
               \sem{\prog{p}},\id] \comp
               \mathrm{dist} \comp \pv{\sem{\prog{b}}}{\id}
        \]
\end{frame}

\begin{frame}{The Least Fixpoint Theorem}
        
        \begin{theorem}
        Every continuous, monotone map $f : X \to X$ has
        a \alert{\underline{least fixpoint}}
        \[
                \mathrm{lfp}\ f = \vee_{n \in \Nats}\, f^n(\bot)
        \]
        \end{theorem}

        \begin{block}{Exercise}
                Prove the theorem
        \end{block}

        \bigskip
        \pause
        And finally \dots
\end{frame}

\begin{frame}{Programs and a denotational semantics}
        \begin{block}{\vspace*{-3.5ex}}
        \begin{center}
        $\prog{p} ::= \mathtt{x} := \mathtt{e} \mid
	\mathtt{p \> {\blue ;} \> p} \mid
	\mathtt{{\blue if} \> b \> {\blue then} \> p \> {\blue else} \> p} \mid
	\mathtt{{\blue while} \> b \> {\blue do} \> \{ \> p \> \}}$
        \end{center}
	\end{block}
        \vspace{-0.5cm}
        \begin{align*}
                \sem{\prog{x : = e}} & = \sigma \mapsto \sigma[\sem{\prog{e}}/\prog{x}] \\[2pt]
                \sem{\prog{p} \> {\blue ;} \> \prog{q}} & 
                = \sem{\prog{q}} \comp \sem{\prog{p}} \\[5pt]
                \sem{\mathtt{{\blue if} \> b \> {\blue then} \> p \> {\blue else} \> q}}
                                                        & 
                                                        = [\sem{\prog{p}},\sem{\prog{q}}] \comp
                                                        \mathrm{dist} \comp \pv{\sem{\prog{b}}}{\id}
                                                        \\[5pt]
                \sem{\mathtt{{\blue while} \> b \> {\blue do} \> \{ \> p \> \}} }
                                                        & \> {=} \> \mathrm{lfp}\ \Big (k \mapsto
                [k \comp 
                \sem{\prog{p}},\id] \comp
                \mathrm{dist} \comp \pv{\sem{\prog{b}}}{\id} \Big )
        \end{align*}
\end{frame}

\begin{frame}{The semantics at Work}
        Prove the following equivalences
        \begin{itemize}
                \item $\mathtt{{\blue while} \> b \> \{ p \} } \equiv
                        \mathtt{{\blue if} \> b \> {\blue then} \> p \> {\blue ;} \>
                        \mathtt{{\blue while} \> b \> \{ p \} } 
                        \> {\blue else} \> skip}$
                \item $\mathtt{{\blue while} \> b \> \{ p \} } \> {\blue ;} \> \mathtt{q} \equiv
                      \mathtt{{\blue if} \> b \> {\blue then} \> p \> {\blue ;}  \> {\blue while} \> b \> \{ p \}
                      \> {\blue ;} \> q
                      \> {\blue else} \> q}$
                \item $\mathtt{{\blue while} \> ff \> \{ p \} \> {\blue ;} \> q } \equiv \mathtt{q}$
                \item $\mathtt{{\blue while} \> tt \> \{ p \} } \equiv
                      \mathtt{{\blue while} \> tt \> \{ q \} }$

        \end{itemize}

        \pause
        \smallskip
        Prove the following implication
        \[
                \sem{\prog{p}} = \sem{\prog{q}} \Longrightarrow
                \text{for all contexts } C. \,
                \sem{C[\prog{p}]} = \sem{C[\prog{q}]}
        \]
\end{frame}

\begin{frame}{The relation between big-step and denotational semantics}

        \begin{theorem}
                $\pv{\prog{p}}{\sigma} \Downarrow \sigma' \text{ iff }
                \sem{\prog{p}}(\sigma) = \sigma'$
        \end{theorem}

        \begin{block}{Proof}
                Yet again \, \dots \, \alert{\underline{induction}}
        \end{block}

        \bigskip
        \begin{corollary}
                $\prog{p} \equiv_o \prog{q} \> \text{ iff } \> \sem{\prog{p}} =
                \sem{\prog{q}}$
        \end{corollary}

        \smallskip
        \begin{corollary}[Holy grail]
                $
                %
                \sem{\prog{p}} = \sem{\prog{q}} \> \text{ iff } \>
                \forall C. \, \sem{C[\prog{p}]} = \sem{C[\prog{q}]} 
                \> \text{ iff } \>
                \prog{p} \equiv \prog{q}
                %
                $
        \end{corollary}

\end{frame}

\section{Conclusions}

\begin{frame}{Conclusions}
        We (\alert{\underline{very}} briefly) studied denotational semantics

        It typically requires more investment than operational counterparts

        \dots \, but this usually pays off

        Numerous extensions to emerging programming languages

        \dots \, quantum, probabilistic, hybrid, stochastic \dots

        \vfill
        Further details in \eg\ \cite[Chapter 2]{reynolds98} and \cite[Chapter
        5]{winskel93}.
\end{frame}

\bibliographystyle{amsalpha}
\bibliography{denotational}

\end{document}

