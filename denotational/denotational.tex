\documentclass{beamer}
\usepackage{etex} % fixes new-dimension error
\usepackage{lmodern}
\usepackage[T1]{fontenc}
\usetheme{metropolis}
\metroset{block=fill}
%\usetheme{Boadilla}

\setbeamertemplate{footline}
{
  \leavevmode%
  \hbox{%
  \begin{beamercolorbox}[wd=.4\paperwidth,ht=2.25ex,dp=1ex,center]{author in head/foot}%
    \usebeamerfont{author in head/foot}\insertshortauthor
  \end{beamercolorbox}%
  \begin{beamercolorbox}[wd=.5\paperwidth,ht=2.25ex,dp=1ex,center]{title in head/foot}%
    \usebeamerfont{title in head/foot}\insertsection
  \end{beamercolorbox}%
  \begin{beamercolorbox}[wd=.1\paperwidth,ht=2.25ex,dp=1ex,right]{date in head/foot}%
    \insertframenumber{} / \inserttotalframenumber\hspace*{2ex} 
  \end{beamercolorbox}}%
  \vskip0pt%
}
\makeatother
\setbeamertemplate{navigation symbols}{}
%----------------------------------------------------------------------------
\usepackage{graphicx,amsmath}
\usepackage{stmaryrd} % cf. interleave
\usepackage{booktabs}
\usepackage{amscd}
\usepackage{multicol}
\usepackage[absolute,overlay]{textpos}
\usepackage{alltt}
\usepackage{proof}
\usepackage{listings}
\usepackage[all]{xy}
\input{macros}
%------ using pstricks (rnode etc) ------------------------------------------
\usepackage{pstricks,pst-node,pst-text,pst-3d}

% ------ using color ---------------------------------------------------------
\newrgbcolor{goldenrod}{.80392 .60784 .11373}
\newrgbcolor{darkgoldenrod}{.5451 .39608 .03137}
\newrgbcolor{brown}{.15 .15 .15}
\newrgbcolor{darkolivegreen}{.33333 .41961 .18431}
\def\gold#1{{\goldenrod #1}}
\def\dgold#1{{\alert{#1}}}
\def\dkb#1{{\blue #1}}
\def\tdkb#1{\textbf{\darkblue #1}}
\def\gre#1{{\darkolivegreen #1}}
\def\gry#1{{\gray #1}}
\def\rdb#1{{\red #1}}
%----------------------------------------------------------------------------

\AtBeginSection[]
{
    \begin{frame}
        \frametitle{Table of Contents}
        \tableofcontents[currentsection]
    \end{frame}
}

\author[Renato Neves]{Renato Neves}

% logos of institutions
\titlegraphic{
  \begin{textblock*}{5cm}(6.5cm,7.4cm)
     \includegraphics[scale=0.06]{images/uminho.png}\hspace*{.85cm}~%
  \end{textblock*}
  \begin{textblock*}{5cm}(9.2cm,7.45cm)
    \includegraphics[scale=0.50]{images/haslab.pdf}
  \end{textblock*}
}

% No date
\date{}

\begin{document}

\title{Denotational Semantics}

\frame[plain]{\titlepage}

\section{Outline}

\begin{frame}{Semantics for every season}

        \hspace*{+5pt}\makebox[\linewidth][c]{%
        \begin{tabular}{ l l }
        Operational semantics & How a program operates
        \\
        \alert{\underline{Denotational semantics}} & What a program is 
        \\
        Axiomatic semantics & Which logical properties a program satisfies
        \end{tabular}
        }

\end{frame}

\section{Motivation}

\begin{frame}{Contextual equivalence}
        We adopted the following notion of
        \alert{\underline{program equivalence}}         
        \begin{flalign*}
                \text{ $\mathtt{p} \equiv_o \mathtt{q}$ iff $\Big ($for
                every $\sigma$. }
                \langle \mathtt{p}, \sigma \rangle \Downarrow \mathtt{\sigma'} 
                \text{ iff } \langle \mathtt{q}, \sigma \rangle \Downarrow \mathtt{\sigma'}
                \Big )
        \end{flalign*}

        \pause
        But compilers adopt the following \alert{\underline{stronger}} version
        \begin{flalign*}
                \text{ $\mathtt{p} \equiv \mathtt{q}$ iff }
                \Big ( \text{for
                every context $C$. }
                C[\mathtt{p}] \equiv_o C[\mathtt{q}]
                \Big )
        \end{flalign*}

        \pause
        \bigskip
        \centering
        \fbox{ Why is that ? }
\end{frame}

\begin{frame}{Contextual equivalence for boolean terms}

        \begin{block}{Boolean contexts}
        \[
                C ::= [-] \mid C \wedge \prog{b} \mid \prog{b} \wedge C \mid \neg C
        \]
        \end{block} 

        \bigskip
        \begin{block}{Exercise}
                Prove the implication $\prog{b_1} \equiv_o \prog{b_2} 
                \Longrightarrow \prog{b_1} \equiv \prog{b_2}$
        \end{block}

        \pause
        \vfill
        Homework: Repeat this exercise for arithmetic expressions
\end{frame}

\begin{frame}{Contextual equivalence for while-programs}

        \begin{block}{Program contexts}
                \small{
                $C ::= [-] \mid C \> \prog{\blue ;} \> \prog{p} 
                \mid \prog{p} \> \prog{\blue ;} \> C \mid
                \mathtt{{\blue if} \> b \> {\blue then} \>} C \mathtt{\> {\blue else} \> p} 
                \mid
                \mathtt{{\blue if} \> b \> {\blue then} \> p \> {\blue else} \>} C 
                \mid 
                \mathtt{{\blue while} \> b \> {\blue do} \> \{ \>} C \mathtt{\> \}}$
        }
        \end{block} 

        \vfill
        \pause
        \centering
        \fbox{Can we still prove $\prog{p} \equiv_o \prog{q} \Longrightarrow
        \prog{p} \equiv \prog{q}$ ?}
\end{frame}

\begin{frame}{Programs as part of a mathematical theory}
        
        \begin{center}
        \fbox{Programming language} \hspace{0.5cm} \scalebox{1.5}{${\hookrightarrow}$} 
        \hspace{0.5cm} \fbox{Mathematical theory}
        \end{center}

        The latter include \emph{e.g.}
        \begin{itemize}
                \item functions (recall program calculus)
                \item linear algebra
                \item relations
                \item games
                \item \alert{\underline{domain theory}}
        \end{itemize}
\end{frame}

\section{My first denotational semantics}

\begin{frame}{Boolean terms and their denotational semantics}
        \begin{block}{\vspace*{-3.5ex}}
        \[
                \prog{b} ::= \prog{x} \mid \prog{b} \wedge \prog{b} \mid \neg \prog{b}
        \]
        \end{block} 

        \vfill
        Terms $\prog{b}$ interpreted as \alert{\underline{functions}}
        $\sem{\prog{b}} : State \to 2$
        \begin{align*}
                \sem{\prog{x}}(\sigma) & = \sigma (x) \\
                \sem{\prog{b_1} \wedge \prog{b_2}} & 
                = (\wedge) \comp \pv{\sem{\prog{b_1}}}{\sem{\prog{b_2}}} \\
                \sem{\neg \prog{b}} & = (\neg) \comp \sem{\prog{b}}
        \end{align*}
\end{frame}

\begin{frame}{Boolean terms and their denotational semantics}
        
        We can now reduce checking for equality of denotations \dots
        \dots\ to \alert{\underline{program calculus}} and \alert{\underline{boolean algebra}}

        \pause
        \begin{align*}
                \sem{\prog{b_1} \wedge \prog{b_2}} & =
                (\wedge) \comp \pv{\sem{\prog{b_1}}}{\sem{\prog{b_2}}}
                \\
                                                   & = (\wedge) \comp \mathrm{sw}
                                                   \comp 
                                                   \pv{\sem{\prog{b_1}}}{\sem{\prog{b_2}}}
                                                   \\
                                                   & = (\wedge) \comp \pv{\pi_2}{\pi_1}
                                                   \comp 
                                                   \pv{\sem{\prog{b_1}}}{\sem{\prog{b_2}}}
                                                   \\
                                                   & = (\wedge) \comp \pv{\pi_2
                                                   \comp \pv{\sem{\prog{b_1}}}{
                                                   \sem{\prog{b_2}}}}{\pi_1
                                                   \comp \pv{\sem{\prog{b_1}}}{\sem{\prog{b_2}}}}
                                                   \\
                                                   & = (\wedge) \comp 
                                                   \pv{\sem{\prog{b_2}}}{\sem{\prog{b_1}}}
                                                   \\
                                                   & = \sem{\prog{b_2} \wedge \prog{b_1}}
                                                   \\[20pt]
              \sem{\prog{b} \wedge \prog{b}} & = \dots \dots
        \end{align*}
\end{frame}

\begin{frame}{Boolean terms and their denotational semantics}

        \begin{theorem}
                For every term $\prog{b}$ and memory $\sigma$ we
                have 
                $\pv{\prog{b}}{\sigma} \Downarrow v \text{ iff }
                \sem{\prog{b}}(\sigma) = v$
        \end{theorem}

        \begin{proof}
                Straightforward \alert{\underline{induction}}
        \end{proof}
\end{frame}

\begin{frame}{Exercises}
        \begin{enumerate}
                \item Define a denotational semantics for arithmetic
                        expressions $\prog{e}$
                \item Prove that $\sem{\prog{e_1} + \prog{e_2}} =
                        \sem{\prog{e_2} + \prog{e_1}}$
                \item Prove the equivalence $\pv{\prog{e}}{\sigma} \Downarrow v
                        \text{ iff } \sem{\prog{e}}(\sigma) = v$
        \end{enumerate}
\end{frame}

\section{Denotational semantics of a while-language}

\begin{frame}{Key takeaways}
        Programs $\prog{p}$ interpreted as \alert{\underline{functions}}
        $\sem{\prog{p}} : State_\bot \to State_\bot$

        $State_\bot = State \cup \{ \bot \}$ where $\bot$ represents
        \alert{\underline{non-termination}}

        Sequential composition will be \alert{\underline{function composition}}
\end{frame}

\begin{frame}{Programs and their denotational semantics}
	\begin{block}{Programs}
        $\prog{p} ::= \mathtt{x} := \mathtt{e} \mid
	\mathtt{p \> {\blue ;} \> p} \mid
	\mathtt{{\blue if} \> b \> {\blue then} \> p \> {\blue else} \> p} \mid
	\mathtt{{\blue while} \> b \> {\blue do} \> \{ \> p \> \}}$
	\end{block}

        \begin{align*}
                \sem{\prog{x : = e}}(\sigma) & = \sigma[\sem{\prog{e}}/\prog{x}] \\
                \sem{\prog{p} \> {\blue ;} \> \prog{q}} & 
                = \sem{\prog{q}} \comp \sem{\prog{p}} \\
                \sem{\mathtt{{\blue if} \> b \> {\blue then} \> p \> {\blue else} \> p}}
                                                        & 
                                                        = [\sem{\prog{p}},\sem{\prog{q}}] \comp
                                                        \mathrm{dist} \comp \pv{\sem{\prog{b}}}{\id}
                \\
                \sem{\mathtt{{\blue while} \> b \> {\blue do} \> \{ \> p \> \}} }
                                                        & =  \dots
%                [\sem{\mathtt{{\blue while} \> b \> {\blue do} \> \{ \> p \> \}} } \comp 
%                \sem{\prog{p}},\id] \comp
%                \mathrm{dist} \comp \pv{\sem{\prog{b}}}{\id}
        \end{align*}

        \pause
\end{frame}

\begin{frame}{Profits!}
        Let us capitalise on this new semantics \dots

        \dots\ recall when we had to prove the two equivalences
        \begin{itemize}
                \item $\mathtt{(p \> {\blue ;} \> q) \> {\blue ;} \> r} \equiv
                        \mathtt{p \> {\blue ;} \> (q \> {\blue ;} \> r)}$
                \item $\mathtt{({\blue if} \> b \> {\blue then} \> p \> {\blue else} \> q) 
                      \> {\blue ;} \> \prog{r}} \equiv 
                      \mathtt{{\blue if} \> b \> {\blue then} \> p \> {\blue ;} \> r \> 
                      {\blue else} \> q \> {\blue ;} \> r}$ 
        \end{itemize}
        with the \alert{\underline{big-step semantics}}

        \bigskip
        Show now that
        \begin{itemize}
                \item $\sem{\mathtt{(p \> {\blue ;} \> q) \> {\blue ;} \> r}} =
                        \sem{\mathtt{p \> {\blue ;} \> (q \> {\blue ;} \> r)}}$
                \item $\sem{\mathtt{({\blue if} \> b \> {\blue then} \> p \> {\blue else} \> q) 
                        \> {\blue ;} \> \prog{r}}} =
                      \sem{\mathtt{{\blue if} \> b \> {\blue then} \> p \> {\blue ;} \> r \> 
                      {\blue else} \> q \> {\blue ;} \> r}}$ 
        \end{itemize}

\end{frame}

\begin{frame}{Programs and their denotational semantics}
	\begin{block}{Programs}
        $\prog{p} ::= \mathtt{x} := \mathtt{e} \mid
	\mathtt{p \> {\blue ;} \> p} \mid
	\mathtt{{\blue if} \> b \> {\blue then} \> p \> {\blue else} \> p} \mid
	\mathtt{{\blue while} \> b \> {\blue do} \> \{ \> p \> \}}$
	\end{block}

        \begin{align*}
                \sem{\prog{x : = e}}(\sigma) & = \sigma[\sem{\prog{e}}/\prog{x}] \\
                \sem{\prog{p} \> {\blue ;} \> \prog{q}} & 
                = \sem{\prog{q}} \comp \sem{\prog{p}} \\
                \sem{\mathtt{{\blue if} \> b \> {\blue then} \> p \> {\blue else} \> p}}
                                                        & 
                                                        = [\sem{\prog{p}},\sem{\prog{q}}] \comp
                                                        \mathrm{dist} \comp \pv{\sem{\prog{b}}}{\id}
                \\
                \sem{\mathtt{{\blue while} \> b \> {\blue do} \> \{ \> p \> \}} }
                                                        & = 
                [\sem{\mathtt{{\blue while} \> b \> {\blue do} \> \{ \> p \> \}} } \comp 
                \sem{\prog{p}},\id] \comp
                \mathrm{dist} \comp \pv{\sem{\prog{b}}}{\id}
        \end{align*}
\end{frame}

\begin{frame}{Programs and their denotational semantics}

        \begin{align*}
                \sem{\prog{x : = e}}(\sigma) & = \sigma[\sem{\prog{e}}/\prog{x}] \\
                \sem{\prog{p} \> {\blue ;} \> \prog{q}} & 
                = \sem{\prog{q}} \comp \sem{\prog{p}} \\
                \sem{\mathtt{{\blue if} \> b \> {\blue then} \> p \> {\blue else} \> p}}
                                                        & 
                                                        = [\sem{\prog{p}},\sem{\prog{q}}] \comp
                                                        \mathrm{dist} \comp \pv{\sem{\prog{b}}}{\id}
                \\
                \sem{\mathtt{{\blue while} \> b \> {\blue do} \> \{ \> p \> \}} }
                                                        & \> {\red \scalebox{1.2}{=}} \>
                [\sem{\mathtt{{\blue while} \> b \> {\blue do} \> \{ \> p \> \}} } \comp 
                \sem{\prog{p}},\id] \comp
                \mathrm{dist} \comp \pv{\sem{\prog{b}}}{\id}
        \end{align*}

        \pause
        \begin{flushright}
        \emph{Brexit means brexit}


        \scriptsize{(Theresa May) \url{https://www.youtube.com/watch?v=oRDfFJAu6Bo}}
        \end{flushright}

\end{frame}

\section{Domain theory}

\bibliographystyle{amsalpha}
\bibliography{big_step}
\end{document}

