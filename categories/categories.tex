\documentclass{beamer}
\usepackage{etex} % fixes new-dimension error
\usepackage{lmodern}
\input{macros/preamble}
\input{macros/macros}
% Categories
\newcommand{\catfont}[1]{\mathsf{#1}}
\newcommand{\Law}{\catfont{Law}}
\newcommand{\catV}{\catfont{V}}
\newcommand{\catX}{\catfont{X}}
\newcommand{\catC}{\catfont{C}}
\newcommand{\catD}{\catfont{D}}
\newcommand{\catA}{\catfont{A}}
\newcommand{\catB}{\catfont{B}}
\newcommand{\catI}{\catfont{I}}
\newcommand{\Set}{\catfont{Set}}
%\newcommand{\Top}{\catfont{Top}}
\newcommand{\Pos}{\catfont{Pos}}
\newcommand{\Inj}{\catfont{Inj}}
\newcommand{\Det}{\catfont{RMhat}}
\newcommand{\CoAlg}[1]{\catfont{CoAlg}\left (#1 \right )}
\newcommand{\Mon}{\catfont{Mon}}
\newcommand{\Mnd}{\catfont{Mnd}(\catC)}
\newcommand{\SMnd}{\catfont{Mnd}(\Set)}
\newcommand{\CLat}{\catfont{CLat}}
\newcommand{\Stone}{\catfont{Stone}}
\newcommand{\Spectral}{\catfont{Spectral}}
\newcommand{\CompHaus}{\catfont{CompHaus}}
\newcommand{\Subs}[2]{\catfont{Sub}_{}}
\newcommand{\Cone}{\catfont{Cone}}
\newcommand{\StComp}{\catfont{StablyComp}}
\newcommand{\PosC}{\catfont{PosComp}}
\newcommand{\Haus}{\catfont{Haus}}
\newcommand{\Meas}{\catfont{Meas}}
\newcommand{\Ord}{\catfont{Ord}}
\newcommand{\TComp}{\catC_{\monadT,S}}
\newcommand{\SetComp}{\Set_{\monadT,S}}
% Functors
\newcommand{\funfont}[1]{#1}
\newcommand{\funF}{\funfont{F}}
\newcommand{\funU}{\funfont{U}}
\newcommand{\funG}{\funfont{G}}
\newcommand{\funT}{\funfont{T}}
\newcommand{\funI}{\funfont{I}}
% Special functors
\newcommand{\sfunfont}[1]{\mathrm{#1}}
\newcommand{\sfunF}{\sfunfont{F}}
\newcommand{\Pow}{\sfunfont{P}}
\newcommand{\NPow}{\sfunfont{Q}}
\newcommand{\Dist}{\sfunfont{D}}
\newcommand{\Maybe}{\sfunfont{M}}
\newcommand{\List}{\sfunfont{L}}
\newcommand{\Forg}[1]{\sfunfont{U}_{#1}}
\newcommand{\UForg}{\sfunfont{U}}
\newcommand{\Id}{\sfunfont{Id}}
\newcommand{\Vie}{\sfunfont{V}}
\newcommand{\Disc}{\funfont{D}}
\newcommand{\Weight}{\sfunfont{W}}
\newcommand{\homf}{\sfunfont{hom}}
\newcommand{\Yoneda}{\sfunfont{Y}}
% Diagram functors
\newcommand{\Diag}{\mathscr{D}}
\newcommand{\KDiag}{\mathscr{K}}
\newcommand{\LDiag}{\mathscr{L}}
% Monads
\newcommand{\monadfont}[1]{\mathbb{#1}}
\newcommand{\monadT}{\monadfont{T}}
\newcommand{\monadS}{\monadfont{S}}
\newcommand{\monadU}{\monadfont{U}}
\newcommand{\monadH}{\monadfont{H}}
% Adjunctions
\newcommand\adjunct[2]{\xymatrix@=8ex{\ar@{}[r]|{\top}\ar@<1mm>@/^2mm/[r]^{{#2}}
& \ar@<1mm>@/^2mm/[l]^{{#1}}}}
\newcommand\adjunctop[2]{\xymatrix@=8ex{\ar@{}[r]|{\bot}\ar@<1mm>@/^2mm/[r]^{{#2}}
& \ar@<1mm>@/^2mm/[l]^{{#1}}}}
% Retractions
\newcommand\retract[2]{\xymatrix@=8ex{\ar@{}[r]|{}\ar@<1mm>@/^2mm/@{^{(}->}[r]^{{#2}}
& \ar@<1mm>@/^2mm/@{->>}[l]^{{#1}}}}
% Special sets
\newcommand{\sconstrfont}[1]{\mathrm{#1}}
\newcommand{\Sone}{\sconstrfont{1}}
\newcommand{\Stwo}{\sconstrfont{2}}
% Sets of numbers
\newcommand{\Nats}{\mathbb{N}}
\newcommand{\Reals}{\mathbb{R}}
%\newcommand{\Rz}{\mathbb{R}_{\geq 0}}
% Constructions
%\newcommand{\pv}[2]{\langle #1, #2 \rangle}
\newcommand{\closure}[1]{\overline{#1}}
\newcommand{\curry}[1]{\overline{#1}}
% Program Operations
%\newcommand{\blue}[1]{\textcolor{blue}{#1}}
\newcommand{\prog}[1]{\mathtt{#1}}
\newcommand{\pskip}{\prog{\blue{skip}}}
\newcommand{\ret}[1]{\prog{\blue{return}} \> #1}
\newcommand{\pcomp}{\prog{ \> \blue{;} \>}}
\newcommand{\pife}[3]{\prog{\blue{if} \>} #1 \prog{\> \blue{then} \> } #2
\prog{\> \blue{else} \>} #3}
\newcommand{\pwhile}[4]{(#1 \leftarrow #2) \pcomp \prog{\blue{while}} \> #3 \> \{ \> #4 \> \} }
\newcommand{\pass}{\prog{\> \blue{:=} \>}}
\newcommand{\lookup}{\prog{\> \blue{lookup_l}}}
\newcommand{\pmlookup}[1]{\prog{\blue{lookup_#1}}}
\newcommand{\slookup}{\prog{\> \blue{lookup}}}
\newcommand{\mwrite}{\prog{\> \blue{write_{(l,v)}} \>}}
\newcommand{\mmwrite}{\prog{\> \blue{write_{(l,v)}} \>}}
\newcommand{\pmwrite}[1]{\prog{\blue{write_{#1}} \>}}

\newcommand{\mswrite}{\prog{\> \blue{write} \>}}
\newcommand{\pOp}[1]{\prog{\blue{#1}}}
% Factorisations
\newcommand{\EClass}{E}
\newcommand{\MClass}{M}
\newcommand{\MConeClass}{\mathcal{M}}
% Misc
\newcommand{\rnote}[1]{\textcolor{MidnightBlue}{[RN: #1]}}
\newcommand{\pfs}{\hspace{0.2cm}}
%\newcommand{\const}[1]{\underline{#1}}
\newcommand{\blank}{\, - \,}
%\newcommand{\comp}{\cdot}
\newcommand{\df}[1]{\emph{\textbf{#1}}}
\newcommand{\colim}{\mathrm{colim}}
\newcommand{\limt}{\mathrm{lim}}
\newcommand{\str}{\mathrm{str}}
\newcommand{\inl}{\mathrm{inl}}
\newcommand{\inr}{\mathrm{inr}}
%\newcommand{\distr}{\mathrm{dist}}
\newcommand{\sem}[1]{\llbracket #1 \rrbracket}
\newcommand{\pullbackcorner}[1][dr]{\save*!/#1+1.2pc/#1:(1,-1)@^{|-}\restore}
\newcommand{\pushoutcorner}[1][dr]{\save*!/#1-1.2pc/#1:(-1,1)@^{|-}\restore}
%\DeclareMathOperator{\img}{\mathsf{im}}
%\DeclareMathOperator{\dom}{\mathsf{dom}}
\DeclareMathOperator{\codom}{\mathsf{codom}}
\DeclareMathOperator{\ev}{\mathsf{ev}}
%\DeclareMathOperator{\id}{\mathsf{id}}
%\DeclareMathOperator{\st}{\mathsf{st}}
% Abbreviations
%\newcommand{\cf}{\emph{cf.}}
%\newcommand{\ie}{\emph{i.e.}}
%\newcommand{\eg}{\emph{e.g.}}
% Textcolors
%\newcommand{\red}[1]{\textcolor{red}{#1}}
% Types
\newcommand{\typefont}[1]{\mathbb{#1}}
\newcommand{\typeOne}{1}
\newcommand{\typeTwo}{2}
\newcommand{\typeA}{\typefont{A}}
\newcommand{\typeB}{\typefont{B}}
\newcommand{\typeC}{\typefont{C}}
\newcommand{\typeV}{\typefont{V}}
\newcommand{\typeD}{\typefont{D}}
% RuleName
\newcommand{\rulename}[1]{\mathrm{#1}}
% Terms
\newcommand{\vljud}{\vdash}
\newcommand{\cojud}{\vdash_{\mathtt{c}}}
\newcommand{\stjud}{\vdash_{\mathtt{w}}}
\newcommand{\vl}{\mathtt{v}}
\newcommand{\co}{\mathtt{c}}

% Terms
\newcommand{\closVal}{\mathsf{Values}(\emptyset, \typeA)}
\newcommand{\closTerm}{\mathsf{Terms}(\emptyset, \typeA)}
\newcommand{\closValP}[1]{\mathsf{Values}(\emptyset, #1)}
\newcommand{\closTermP}[1]{\mathsf{Terms}(\emptyset, #1)}

% Components
%\newcommand{\bcomp}{\prog{\blue{c}}}

% Contextual equivalence
\newcommand{\ctxeq}{\simeq_{\prog{ctx}}}

\newcommand{\sigt}{sig$_\monadT$}


%-------------- template --------------------------------------------------
\usetheme{metropolis}
\metroset{block=fill}
%\usetheme{Boadilla}
\setbeamertemplate{footline}
{
  \leavevmode%
  \hbox{%
  \begin{beamercolorbox}[wd=.4\paperwidth,ht=2.25ex,dp=1ex,center]{author in head/foot}%
    \usebeamerfont{author in head/foot}\insertshortauthor
  \end{beamercolorbox}%
  \begin{beamercolorbox}[wd=.5\paperwidth,ht=2.25ex,dp=1ex,center]{title in head/foot}%
    \usebeamerfont{title in head/foot}\insertsection
  \end{beamercolorbox}%
  \begin{beamercolorbox}[wd=.1\paperwidth,ht=2.25ex,dp=1ex,right]{date in head/foot}%
    \insertframenumber{} / \inserttotalframenumber\hspace*{2ex} 
  \end{beamercolorbox}}%
  \vskip0pt%
}
\makeatother
\setbeamertemplate{navigation symbols}{}

%----------------------------------------------------------------------------
\usepackage{graphicx,amsmath}
\usepackage{stmaryrd} % cf. interleave
\usepackage{booktabs}
\usepackage{amscd}
\usepackage{multicol}
\usepackage[absolute,overlay]{textpos}
\usepackage{alltt}
\usepackage{proof}
%------ using xy ------------------------------------------------------------
\usepackage[all]{xy}
%\def\larrow#1#2#3{\xymatrix{ #3 & #1 \ar[l] _-{#2} }}
\def\larrow#1#2#3{\xymatrix{ #3 & #1 \ar[l] _--{#2} }}
\def\rarrow#1#2#3{\xymatrix{ #1 \ar[r]^-{#2} & #3 }}
\def\arLaw#1#2#3#4#5{
\xymatrix{
        #1      \ar@/^1pc/[rr]^-{#4} &
        #5 &
        #2      \ar@/^1pc/[ll]^-{#3}
}}
\def\arLeq#1#2#3#4{\arLaw{#1}{#2}{#3}{#4}\leq}
%------ using pstricks (rnode etc) ------------------------------------------
\usepackage{pstricks,pst-node,pst-text,pst-3d}
%------ using color ---------------------------------------------------------

\newrgbcolor{goldenrod}{.80392 .60784 .11373}
\newrgbcolor{darkgoldenrod}{.5451 .39608 .03137}
\newrgbcolor{brown}{.15 .15 .15}
\newrgbcolor{darkolivegreen}{.33333 .41961 .18431}
%
%
\def\gold#1{{\goldenrod #1}}
% \def\dgold#1{{\darkgoldenrod #1}}
\def\dgold#1{{\alert{#1}}}
%\def\brw#1{{\brown #1}}
\def\dkb#1{{\blue #1}}
\def\tdkb#1{\textbf{\darkblue #1}}
%%\def\gre#1{{\green #1}}
\def\gre#1{{\darkolivegreen #1}}
\def\gry#1{{\gray #1}}
\def\rdb#1{{\red #1}}
\def\st{\mathbf{.}\,}
\def\laplace#1#2{*\txt{\mbox{ \fcolorbox{black}{myGray}{$\begin{array}{c}\mbox{#1}\\\\#2\\\\\end{array}$} }}}
%\newcommand{\galois}[2]{#1\; \dashv\; #2}

\def\eqm{\mathbin{\equiv}}                     
\def\noeqm{\mathbin{\not\!\equiv}}  
%\newcommand{\flam}[2]{\lambda_{#1}\; .\; #2}
\def\existential#1#2{\exists_{#1}\;.\; #2}
\def\existencial#1#2{\exists_{#1}\;.\; #2}

\def\pv#1#2{\langle #1, #2 \rangle}
\def\nc#1#2{[#1]#2}
\def\pvo#1#2{\langle \! \! \! \langle #1 \rangle \! \! \! \rangle\, #2}
\def\nco#1#2{\llbracket #1 \rrbracket #2}
\def\cvg#1{\llbracket \downarrow \rrbracket #1}
\def\cvgr#1#2{\llbracket #1 \downarrow \rrbracket #2}
\def\cvgl#1#2{\llbracket \downarrow  #1 \rrbracket #2}
\def\cvglr#1#2{\llbracket \downarrow  #1 \downarrow \rrbracket #2}
\def\lfp#1#2{\mu {#1}\, .\, {#2}}
\def\lpf#1#2{\mu {#1}\, .\, {#2}}
\def\gfp#1#2{\nu {#1}\, .\, {#2}}
\def\gpf#1#2{\nu {#1}\, .\, {#2}}
\def\mset#1{\vvv #1 \vvv}
\def\vvv{\vert \! \vert}
\def\mnc#1{\vvv [#1] \vvv}
\def\mpv#1{\vvv \langle #1 \rangle \vvv}
\def\bcomp#1{#1^{\text{c}}}
\def\eqm{\mathbin{\simeq}}
\def\noeqm{\mathbin{\not\!\simeq}}
\def\universal#1#2{\forall_{#1}\;.\; #2}
\def\existential#1#2{\exists_{#1}\;.\; #2}
\def\oexistential#1#2{\exists^{1}_{#1}\;.\; #2}
\def\MM{\mathcal{M}}
\def\uppaal{\textsc{Uppaal}}
\def\cc#1{\mathcal{C}(#1)}
\def\R{\mathcal{R}}
\def\TL#1{\mathcal{T}(#1)}
\def\HL#1{\mathcal{H}(#1)}
\def\ET#1{\mathsf{ExecTime(#1)}}
\newcommand\mdoubleplus{\mathbin{+\mkern-10mu+}}

\AtBeginSection[]
{
    \begin{frame}
        \frametitle{Table of Contents}
        \tableofcontents[currentsection]
    \end{frame}
}

% context
\AtBeginSection[]
{
    \begin{frame}
        \frametitle{Table of Contents}
        \tableofcontents[currentsection]
    \end{frame}
}
\author[Renato Neves]{Renato Neves}

% logos of institutions
\titlegraphic{
  \begin{textblock*}{5cm}(6.7cm,7.57cm)
     \includegraphics[scale=0.05525]{./images/uminho.png}
  \end{textblock*}
  \begin{textblock*}{5cm}(9.4cm,7.57cm)
    \includegraphics[scale=0.50]{./images/haslab.pdf}
  \end{textblock*}
}

% No date
\date{}


\begin{document}

\title{Buh !}
\frame[plain]{\titlepage}


\title{Next Step: Category Theory}

\frame[plain]{\titlepage}

\section{Basic Notions}

\begin{frame}{In a Nutshell}

        The Mathematics of Mathematics

        \bigskip
        \pause
        Abstract non-sense

        \bigskip
        \pause
        \alert{\underline{\emph{El Libertadora}}}

        \bigskip
        \pause
        You have actually met her before
\end{frame}

\begin{frame}{A Category}

        Consists of a `set' of things (a.k.a. \alert{\underline{objects}}) $A,B,C$ \dots

        \medskip
        arrows (a.k.a. \alert{\underline{morphisms}}) $f : A \to B$ connecting these things

        \medskip
        and an `identity' arrow $\id : A \to A$ for each thing $A$

        \vfill
        \pause
        Arrows are \alert{\underline{composable}} \ie\
        \[
                \infer{
                        g \comp f : A \to C
                }{
                        f : A \to B \qquad g : B \to C}
        \]
        \pause
        (composition is associative and has $\id$ as neutral element)
\end{frame}

\begin{frame}{Examples}
        \begin{itemize}
                \item Sets and functions ($\Set$) \\[10pt]
                \item Posets and monotone functions ($\Pos$) \\[10pt]
                \item Vector spaces and linear functions ($\CVec$) \\[10pt]
                \item Natural numbers and matrices ($\Mat$) \\[10pt]
                \item Reals numbers with infinity and $\leq$ relation ($\CR$)  \\[10pt]
        \end{itemize}
\end{frame}


\begin{frame}{Finality}
        \begin{definition}
           An obj. $X$ is \alert{\underline{final}} if for every obj. $A$ there exists a
           \alert{\underline{unique}} arrow
           \[
                   ! : A \to X
           \]
        \end{definition}
        
        \pause
        \medskip
        In $\Set$, $\Pos$, $\CVec$ it is singleton set
        (represents `\alert{\underline{discarding}} data')
        
        \medskip
        In $\Mat$ it is $0$ (discarding analogy still applies)

        \medskip
        In $\CR$ it is infinity (the greatest element)
\end{frame}


\begin{frame}{Categorical Products}
        \begin{definition}
        An obj. $X$ called \alert{\underline{product}} of objs. $A$ and $B$ if
        \[
               \xymatrix@C=50pt{
                       & C \ar[dl]_{f} \ar[dr]^{g} \ar@{.>}[d]|{ \pv{f}{g} } & 
                       \\
                       A & \ar[l]^{\pi_1} \ar[r]_{\pi_2}
                       X & B
               }
        \]
        \end{definition}

        \pause
        \medskip
        In $\Set$, $\Pos$, $\CVec$ given by Cartesian product

        \medskip
        In $\Mat$ given by addition (leads to vertical stacking of matrices)

        \medskip
        In $\CR$ given by minimum  (\ie\ the greatest lower bound)
\end{frame}


\begin{frame}{Categorical Exponentials}

        \begin{definition}
                An obj. $X$ called \alert{\underline{exponential}} of objs. $A$
                and $B$ if 
        \[
               \xymatrix@C=50pt{
                       A \times B \ar[dr]^{f} 
                       \ar@{.>}[d]|{(\lambda f) \times \id}
                       \\
                       X \times B \ar[r]_{\mathrm{app}} & C
               }
        \]
        \end{definition}

        \pause
        \medskip
        In $\Set$, $\Pos$ given by the set of (monotone) functions $A \to B$

        \pause
        \medskip
        Other examples will be analysed later on
\end{frame}

\section{Lambda-calculus and Cartesian-closed Categories}

\begin{frame}{Cartesian-closed Categories}

       We will now \alert{\underline{vastly generalise}} our semantics of
       $\lambda$-calculus \dots\

       \medskip
       \pause
       \dots\ via the so-called Cartesian-closed categories

       \pause
       \vfill
       \begin{definition}
               A category is called \alert{\underline{Cartesian-closed}} if it
               has a final object, binary products, and exponentials
       \end{definition}
\end{frame}

\begin{frame}{Denotational Semantics in a Cartesian-closed Category}
  Types $\typeA$ interpreted as \alert{\underline{objects}} $\sem{\typeA}$
  \begin{flalign*}
    \sem{1} & = 1 \\
    \sem{\typeA \times \typeB} & = \sem{\typeA} \times \sem{\typeB} \\
   \sem{\typeA \to \typeB} & = \sem{\typeB}^{\sem{\typeA}} 
 \end{flalign*}

 Typing contexts $\Gamma$ interpreted as products
 \begin{flalign*}
  \sem{\Gamma}\ = \sem{x_1 : \typeA_1, \dots, x_n : \typeA_n}\ =
 \sem{\typeA_1}\ \times \dots \times \sem{\typeA_n} 
 \end{flalign*}

 $\lambda$-terms $\Gamma \vljud t : \typeA$ interpreted as \alert{\underline{arrows}}
 \begin{flalign*}
   \sem{\Gamma \vljud t : \typeA}\ : \sem{\Gamma}\ \longrightarrow \sem{\typeA}
 \end{flalign*}

\end{frame}

\begin{frame}{Denotational Semantics in a Cartesian-closed Category}
 \small{
   \begin{flalign*}
      \infer[]{\sem{\Gamma \vljud x_i : \typeA} = \pi_i}{x_i :
        \typeA \in \Gamma} \hspace{1.2cm}
      \infer[]{\sem{\Gamma \vljud \ast : 1}\ =\ !}{}  \hspace{1.2cm}
      \infer[]{\sem{\Gamma \vljud \pi_1 t : \typeA} = \pi_1 \comp f }
      {\sem{\Gamma \vljud t : \typeA \times \typeB} = f}
  \end{flalign*}
  \begin{flalign*}
      \infer[]{\sem{\Gamma \vljud \langle t,s \rangle :
\typeA \times \typeB}\ = \pv{f}{g}}
    {\sem{\Gamma \vljud t: \typeA}\ = f \qquad \sem{\Gamma \vljud s: \typeB}\ = g}
    \hspace{0.3cm}
     \infer[]{\sem{\Gamma \vljud \lambda x : \typeA . \, t : \typeA
        \to \typeB}\ = \lambda f}
      {\sem{\Gamma, x : \typeA \vljud t : \typeB}\ = f}
    \end{flalign*}
   \begin{flalign*}
           \infer[]{\sem{\Gamma \vljud t \, s : \typeB}\ = \mathrm{app} \comp \pv{f}{g}}
      {\sem{\Gamma \vljud  t : \typeA \to \typeB}\ = f \quad
        \sem{\Gamma \vljud  s : \typeA}\ = g}
    \end{flalign*}
}
\end{frame}

\begin{frame}{Case-study: The Boolean Category}
        \[
                \xymatrix@C=40pt{
                        0 \ar[r]^{\leq} \ar@(ul,dl)[]_{\leq} & 1 \ar@(ur,dr)[]^{\leq} 
                }
        \]

        \bigskip
        \bigskip
        Show that it has final object and binary products

        Show that it has exponentials \hspace{6cm}
        {\small (hint: use negation and disjunction)}

        Show that $b \to c \leq (a \to b) \to (a \to c)$
\end{frame}

\begin{frame}{Case-study: Programming to the Rescue !}

        Previous problem \alert{\underline{proved}} via the \alert{\underline{program}}
        \[
                f : \typeB \to \typeC \vljud
                \lambda g. \, \lambda x. \, f(g \, (x)) : (\typeA \to \typeB)
                \to (\typeA \to \typeC)
        \]
        \pause
        Much simpler than before !!!

        \pause
        \bigskip
        But wait \dots\ is then $\lambda$-calculus really a programming language ?
\end{frame}

\section{From Cartesian to Linear}

\begin{frame}{Monoidal Categories}

        A monoidal category allows us
        \begin{itemize}
                \item to \alert{\underline{pair}} two objs. $A$ and $B$ into a single one $A
                        \otimes B$ 
                \item and analogously for morphisms, \ie\
                        \[
                                \infer{
                                        f \otimes g : A \otimes B \to A' \otimes B'
                                }{
                                        f : A \to A' \qquad g : B \to B'
                                }
                        \]
                \item It has an obj. $I$ such that $i : I \otimes A \stackrel{\cong}{\to} A$ for
                        every obj. $A$
        \end{itemize}

        \medskip
        A number of coherence laws hold \dots

        Also assume the existence of a \alert{\underline{symmetry map}}
        $\mathrm{sw} : A \otimes B \stackrel{\cong}{\to} B \otimes A$
\end{frame}


\begin{frame}{Examples}
        \begin{itemize}
                \item Every category with binary products and final object is monoidal  \\[10pt]
                \item $\CVec$ with the \alert{\underline{tensor product}}
                        (the mathematical basis of entanglement) is monoidal
                        \\[10pt]
                \item $\CR$ with addition is monoidal  \\[10pt]
        \end{itemize}
\end{frame}

\begin{frame}{Categorical Exponentials Revisited}

        \begin{definition}
                An obj. $X$ called \alert{\underline{exponential}} of objs. $A$
                and $B$ if 
        \[
               \xymatrix@C=50pt{
                       A \otimes B \ar[dr]^{f} 
                       \ar@{.>}[d]|{(\lambda f) \otimes \id}
                       \\
                       X \otimes B \ar[r]_{\mathrm{app}} & C
               }
        \]
        \end{definition}

        \pause
        \medskip
        In $\Set$, $\Pos$ given by the set of (monotone) functions $A \to B$

        \pause
        \medskip
        In $\CVec$ given by linear maps
\end{frame}

\begin{frame}{Monoidal-closed Categories}

       We will now (finally) give semantics to linear
       $\lambda$-calculus \dots\

       \medskip
       \pause
       \dots\ via the so-called monoidal-closed categories

       \pause
       \vfill
       \begin{definition}
               A category called \alert{\underline{monoidal-closed}} if 
               it is monoidal and has exponentials
       \end{definition}
\end{frame}

\begin{frame}{Denotational Semantics in a Monoidal-closed Category}
  Types $\typeA$ interpreted as \alert{\underline{objects}} $\sem{\typeA}$
  \begin{flalign*}
    \sem{\typeI} & = I \\
    \sem{\typeA \otimes \typeB} & = \sem{\typeA}\ \otimes\ \sem{\typeB} \\
   \sem{\typeA \multimap \typeB} & = \sem{\typeB}^{\sem{\typeA}} 
 \end{flalign*}

 Typing contexts $\Gamma$ interpreted as products
 \begin{flalign*}
  \sem{\Gamma}\ = \sem{x_1 : \typeA_1, \dots, x_n : \typeA_n}\ =
 \sem{\typeA_1}\ \otimes \dots \otimes \sem{\typeA_n} 
 \end{flalign*}

 $\lambda$-terms $\Gamma \vljud t : \typeA$ interpreted as \alert{\underline{arrows}}
 \begin{flalign*}
   \sem{\Gamma \vljud t : \typeA}\ : \sem{\Gamma}\ \longrightarrow \sem{\typeA}
 \end{flalign*}

\end{frame}

\begin{frame}{Denotational Semantics in a Monoidal-closed Category}
  
  \small{
  \begin{flalign*}
      \hspace{-0.2cm}
      \infer[]{\sem{x : \typeA \vljud x : \typeA} = \id}{} \hspace{1cm}
      \infer[]{\sem{(-) \vljud  \ast : \typeI} = \id}{} 
  \end{flalign*}
  \begin{flalign*}
      \hspace{-0.2cm}
      \infer{\sem{\Gamma,\Delta \vljud t \text{ to } \ast.\  \, s: \typeA}\ = i \comp (f \otimes g)}{
              \sem{\Gamma \vljud t : \typeI}\ = f \qquad
              \sem{\Delta \vljud s : \typeA}\ = g
      }
      \hspace{1cm}
      \infer[]{\sem{\Gamma,\Delta \vljud t \otimes s : \typeA \otimes \typeB}\ = f \otimes g}
      {\sem{\Gamma \vljud t : \typeA}\ = f 
      \qquad \sem{\Delta \vljud s: \typeB}\ = g
      }
  \end{flalign*}
  \begin{flalign*}
      \infer{
              \sem{\Gamma, \Delta \vljud 
              \text{pm } t \text{ to } x \otimes y.\, s: \typeC}\ = 
              g \comp \mathrm{sw} \comp (f \otimes \id)
      }{
              \sem{\Gamma \vljud t : \typeA \otimes \typeB}\ = f
              \qquad
              \sem{\Delta, x: \typeA, y: \typeB \vljud s: \typeC}\ = g
      }
  \end{flalign*}
  \begin{flalign*}
      \hspace{-0.2cm}
      \infer[]{\sem{\Gamma \vljud \lambda x : \typeA.\ t : \typeA \multimap \typeB}\ = \lambda f}
      {
              \sem{\Gamma, x: \typeA \vljud t : \typeB}\ = f
      }
      \hspace{1cm}
      \infer[]{
              \sem{\Gamma,\Delta \vljud t \, s : \typeB}\ = \mathrm{app} \comp (f \otimes g)
      }
      {
              \sem{\Gamma \vljud t: \typeA \multimap \typeB}\ = f  \quad
              \sem{\Delta \vljud s:  \typeA}\ = g }
  \end{flalign*}
        } 
  
      \pause
      \begin{center}
              {\scriptsize (Semantics slightly oversimplified)}
      \end{center}
\end{frame}




\end{document}
