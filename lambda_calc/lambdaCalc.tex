\documentclass{beamer}
\usepackage{etex} % fixes new-dimension error
\usepackage{lmodern}
\input{macros/preamble}
\input{macros/macros}
% Categories
\newcommand{\catfont}[1]{\mathsf{#1}}
\newcommand{\Law}{\catfont{Law}}
\newcommand{\catV}{\catfont{V}}
\newcommand{\catX}{\catfont{X}}
\newcommand{\catC}{\catfont{C}}
\newcommand{\catD}{\catfont{D}}
\newcommand{\catA}{\catfont{A}}
\newcommand{\catB}{\catfont{B}}
\newcommand{\catI}{\catfont{I}}
\newcommand{\Set}{\catfont{Set}}
%\newcommand{\Top}{\catfont{Top}}
\newcommand{\Pos}{\catfont{Pos}}
\newcommand{\Inj}{\catfont{Inj}}
\newcommand{\Det}{\catfont{RMhat}}
\newcommand{\CoAlg}[1]{\catfont{CoAlg}\left (#1 \right )}
\newcommand{\Mon}{\catfont{Mon}}
\newcommand{\Mnd}{\catfont{Mnd}(\catC)}
\newcommand{\SMnd}{\catfont{Mnd}(\Set)}
\newcommand{\CLat}{\catfont{CLat}}
\newcommand{\Stone}{\catfont{Stone}}
\newcommand{\Spectral}{\catfont{Spectral}}
\newcommand{\CompHaus}{\catfont{CompHaus}}
\newcommand{\Subs}[2]{\catfont{Sub}_{}}
\newcommand{\Cone}{\catfont{Cone}}
\newcommand{\StComp}{\catfont{StablyComp}}
\newcommand{\PosC}{\catfont{PosComp}}
\newcommand{\Haus}{\catfont{Haus}}
\newcommand{\Meas}{\catfont{Meas}}
\newcommand{\Ord}{\catfont{Ord}}
\newcommand{\TComp}{\catC_{\monadT,S}}
\newcommand{\SetComp}{\Set_{\monadT,S}}
% Functors
\newcommand{\funfont}[1]{#1}
\newcommand{\funF}{\funfont{F}}
\newcommand{\funU}{\funfont{U}}
\newcommand{\funG}{\funfont{G}}
\newcommand{\funT}{\funfont{T}}
\newcommand{\funI}{\funfont{I}}
% Special functors
\newcommand{\sfunfont}[1]{\mathrm{#1}}
\newcommand{\sfunF}{\sfunfont{F}}
\newcommand{\Pow}{\sfunfont{P}}
\newcommand{\NPow}{\sfunfont{Q}}
\newcommand{\Dist}{\sfunfont{D}}
\newcommand{\Maybe}{\sfunfont{M}}
\newcommand{\List}{\sfunfont{L}}
\newcommand{\Forg}[1]{\sfunfont{U}_{#1}}
\newcommand{\UForg}{\sfunfont{U}}
\newcommand{\Id}{\sfunfont{Id}}
\newcommand{\Vie}{\sfunfont{V}}
\newcommand{\Disc}{\funfont{D}}
\newcommand{\Weight}{\sfunfont{W}}
\newcommand{\homf}{\sfunfont{hom}}
\newcommand{\Yoneda}{\sfunfont{Y}}
% Diagram functors
\newcommand{\Diag}{\mathscr{D}}
\newcommand{\KDiag}{\mathscr{K}}
\newcommand{\LDiag}{\mathscr{L}}
% Monads
\newcommand{\monadfont}[1]{\mathbb{#1}}
\newcommand{\monadT}{\monadfont{T}}
\newcommand{\monadS}{\monadfont{S}}
\newcommand{\monadU}{\monadfont{U}}
\newcommand{\monadH}{\monadfont{H}}
% Adjunctions
\newcommand\adjunct[2]{\xymatrix@=8ex{\ar@{}[r]|{\top}\ar@<1mm>@/^2mm/[r]^{{#2}}
& \ar@<1mm>@/^2mm/[l]^{{#1}}}}
\newcommand\adjunctop[2]{\xymatrix@=8ex{\ar@{}[r]|{\bot}\ar@<1mm>@/^2mm/[r]^{{#2}}
& \ar@<1mm>@/^2mm/[l]^{{#1}}}}
% Retractions
\newcommand\retract[2]{\xymatrix@=8ex{\ar@{}[r]|{}\ar@<1mm>@/^2mm/@{^{(}->}[r]^{{#2}}
& \ar@<1mm>@/^2mm/@{->>}[l]^{{#1}}}}
% Special sets
\newcommand{\sconstrfont}[1]{\mathrm{#1}}
\newcommand{\Sone}{\sconstrfont{1}}
\newcommand{\Stwo}{\sconstrfont{2}}
% Sets of numbers
\newcommand{\Nats}{\mathbb{N}}
\newcommand{\Reals}{\mathbb{R}}
%\newcommand{\Rz}{\mathbb{R}_{\geq 0}}
% Constructions
%\newcommand{\pv}[2]{\langle #1, #2 \rangle}
\newcommand{\closure}[1]{\overline{#1}}
\newcommand{\curry}[1]{\overline{#1}}
% Program Operations
%\newcommand{\blue}[1]{\textcolor{blue}{#1}}
\newcommand{\prog}[1]{\mathtt{#1}}
\newcommand{\pskip}{\prog{\blue{skip}}}
\newcommand{\ret}[1]{\prog{\blue{return}} \> #1}
\newcommand{\pcomp}{\prog{ \> \blue{;} \>}}
\newcommand{\pife}[3]{\prog{\blue{if} \>} #1 \prog{\> \blue{then} \> } #2
\prog{\> \blue{else} \>} #3}
\newcommand{\pwhile}[4]{(#1 \leftarrow #2) \pcomp \prog{\blue{while}} \> #3 \> \{ \> #4 \> \} }
\newcommand{\pass}{\prog{\> \blue{:=} \>}}
\newcommand{\lookup}{\prog{\> \blue{lookup_l}}}
\newcommand{\pmlookup}[1]{\prog{\blue{lookup_#1}}}
\newcommand{\slookup}{\prog{\> \blue{lookup}}}
\newcommand{\mwrite}{\prog{\> \blue{write_{(l,v)}} \>}}
\newcommand{\mmwrite}{\prog{\> \blue{write_{(l,v)}} \>}}
\newcommand{\pmwrite}[1]{\prog{\blue{write_{#1}} \>}}

\newcommand{\mswrite}{\prog{\> \blue{write} \>}}
\newcommand{\pOp}[1]{\prog{\blue{#1}}}
% Factorisations
\newcommand{\EClass}{E}
\newcommand{\MClass}{M}
\newcommand{\MConeClass}{\mathcal{M}}
% Misc
\newcommand{\rnote}[1]{\textcolor{MidnightBlue}{[RN: #1]}}
\newcommand{\pfs}{\hspace{0.2cm}}
%\newcommand{\const}[1]{\underline{#1}}
\newcommand{\blank}{\, - \,}
%\newcommand{\comp}{\cdot}
\newcommand{\df}[1]{\emph{\textbf{#1}}}
\newcommand{\colim}{\mathrm{colim}}
\newcommand{\limt}{\mathrm{lim}}
\newcommand{\str}{\mathrm{str}}
\newcommand{\inl}{\mathrm{inl}}
\newcommand{\inr}{\mathrm{inr}}
%\newcommand{\distr}{\mathrm{dist}}
\newcommand{\sem}[1]{\llbracket #1 \rrbracket}
\newcommand{\pullbackcorner}[1][dr]{\save*!/#1+1.2pc/#1:(1,-1)@^{|-}\restore}
\newcommand{\pushoutcorner}[1][dr]{\save*!/#1-1.2pc/#1:(-1,1)@^{|-}\restore}
%\DeclareMathOperator{\img}{\mathsf{im}}
%\DeclareMathOperator{\dom}{\mathsf{dom}}
\DeclareMathOperator{\codom}{\mathsf{codom}}
\DeclareMathOperator{\ev}{\mathsf{ev}}
%\DeclareMathOperator{\id}{\mathsf{id}}
%\DeclareMathOperator{\st}{\mathsf{st}}
% Abbreviations
%\newcommand{\cf}{\emph{cf.}}
%\newcommand{\ie}{\emph{i.e.}}
%\newcommand{\eg}{\emph{e.g.}}
% Textcolors
%\newcommand{\red}[1]{\textcolor{red}{#1}}
% Types
\newcommand{\typefont}[1]{\mathbb{#1}}
\newcommand{\typeOne}{1}
\newcommand{\typeTwo}{2}
\newcommand{\typeA}{\typefont{A}}
\newcommand{\typeB}{\typefont{B}}
\newcommand{\typeC}{\typefont{C}}
\newcommand{\typeV}{\typefont{V}}
\newcommand{\typeD}{\typefont{D}}
% RuleName
\newcommand{\rulename}[1]{\mathrm{#1}}
% Terms
\newcommand{\vljud}{\vdash}
\newcommand{\cojud}{\vdash_{\mathtt{c}}}
\newcommand{\stjud}{\vdash_{\mathtt{w}}}
\newcommand{\vl}{\mathtt{v}}
\newcommand{\co}{\mathtt{c}}

% Terms
\newcommand{\closVal}{\mathsf{Values}(\emptyset, \typeA)}
\newcommand{\closTerm}{\mathsf{Terms}(\emptyset, \typeA)}
\newcommand{\closValP}[1]{\mathsf{Values}(\emptyset, #1)}
\newcommand{\closTermP}[1]{\mathsf{Terms}(\emptyset, #1)}

% Components
%\newcommand{\bcomp}{\prog{\blue{c}}}

% Contextual equivalence
\newcommand{\ctxeq}{\simeq_{\prog{ctx}}}

\newcommand{\sigt}{sig$_\monadT$}


%-------------- template --------------------------------------------------
\usetheme{metropolis}
\metroset{block=fill}
%\usetheme{Boadilla}
\setbeamertemplate{footline}
{
  \leavevmode%
  \hbox{%
  \begin{beamercolorbox}[wd=.4\paperwidth,ht=2.25ex,dp=1ex,center]{author in head/foot}%
    \usebeamerfont{author in head/foot}\insertshortauthor
  \end{beamercolorbox}%
  \begin{beamercolorbox}[wd=.5\paperwidth,ht=2.25ex,dp=1ex,center]{title in head/foot}%
    \usebeamerfont{title in head/foot}\insertsection
  \end{beamercolorbox}%
  \begin{beamercolorbox}[wd=.1\paperwidth,ht=2.25ex,dp=1ex,right]{date in head/foot}%
    \insertframenumber{} / \inserttotalframenumber\hspace*{2ex} 
  \end{beamercolorbox}}%
  \vskip0pt%
}
\makeatother
\setbeamertemplate{navigation symbols}{}

%----------------------------------------------------------------------------
\usepackage{graphicx,amsmath}
\usepackage{stmaryrd} % cf. interleave
\usepackage{booktabs}
\usepackage{amscd}
\usepackage{multicol}
\usepackage[absolute,overlay]{textpos}
\usepackage{alltt}
\usepackage{proof}
%------ using xy ------------------------------------------------------------
\usepackage[all]{xy}
%\def\larrow#1#2#3{\xymatrix{ #3 & #1 \ar[l] _-{#2} }}
\def\larrow#1#2#3{\xymatrix{ #3 & #1 \ar[l] _--{#2} }}
\def\rarrow#1#2#3{\xymatrix{ #1 \ar[r]^-{#2} & #3 }}
\def\arLaw#1#2#3#4#5{
\xymatrix{
        #1      \ar@/^1pc/[rr]^-{#4} &
        #5 &
        #2      \ar@/^1pc/[ll]^-{#3}
}}
\def\arLeq#1#2#3#4{\arLaw{#1}{#2}{#3}{#4}\leq}
%------ using pstricks (rnode etc) ------------------------------------------
\usepackage{pstricks,pst-node,pst-text,pst-3d}
%------ using color ---------------------------------------------------------

\newrgbcolor{goldenrod}{.80392 .60784 .11373}
\newrgbcolor{darkgoldenrod}{.5451 .39608 .03137}
\newrgbcolor{brown}{.15 .15 .15}
\newrgbcolor{darkolivegreen}{.33333 .41961 .18431}
%
%
\def\gold#1{{\goldenrod #1}}
% \def\dgold#1{{\darkgoldenrod #1}}
\def\dgold#1{{\alert{#1}}}
%\def\brw#1{{\brown #1}}
\def\dkb#1{{\blue #1}}
\def\tdkb#1{\textbf{\darkblue #1}}
%%\def\gre#1{{\green #1}}
\def\gre#1{{\darkolivegreen #1}}
\def\gry#1{{\gray #1}}
\def\rdb#1{{\red #1}}
\def\st{\mathbf{.}\,}
\def\laplace#1#2{*\txt{\mbox{ \fcolorbox{black}{myGray}{$\begin{array}{c}\mbox{#1}\\\\#2\\\\\end{array}$} }}}
%\newcommand{\galois}[2]{#1\; \dashv\; #2}

\def\eqm{\mathbin{\equiv}}                     
\def\noeqm{\mathbin{\not\!\equiv}}  
%\newcommand{\flam}[2]{\lambda_{#1}\; .\; #2}
\def\existential#1#2{\exists_{#1}\;.\; #2}
\def\existencial#1#2{\exists_{#1}\;.\; #2}

\def\pv#1#2{\langle #1 \rangle #2}
\def\nc#1#2{[#1]#2}
\def\pvo#1#2{\langle \! \! \! \langle #1 \rangle \! \! \! \rangle\, #2}
\def\nco#1#2{\llbracket #1 \rrbracket #2}
\def\cvg#1{\llbracket \downarrow \rrbracket #1}
\def\cvgr#1#2{\llbracket #1 \downarrow \rrbracket #2}
\def\cvgl#1#2{\llbracket \downarrow  #1 \rrbracket #2}
\def\cvglr#1#2{\llbracket \downarrow  #1 \downarrow \rrbracket #2}
\def\lfp#1#2{\mu {#1}\, .\, {#2}}
\def\lpf#1#2{\mu {#1}\, .\, {#2}}
\def\gfp#1#2{\nu {#1}\, .\, {#2}}
\def\gpf#1#2{\nu {#1}\, .\, {#2}}
\def\mset#1{\vvv #1 \vvv}
\def\vvv{\vert \! \vert}
\def\mnc#1{\vvv [#1] \vvv}
\def\mpv#1{\vvv \langle #1 \rangle \vvv}
\def\bcomp#1{#1^{\text{c}}}
\def\eqm{\mathbin{\simeq}}
\def\noeqm{\mathbin{\not\!\simeq}}
\def\universal#1#2{\forall_{#1}\;.\; #2}
\def\existential#1#2{\exists_{#1}\;.\; #2}
\def\oexistential#1#2{\exists^{1}_{#1}\;.\; #2}
\def\MM{\mathcal{M}}
\def\uppaal{\textsc{Uppaal}}
\def\cc#1{\mathcal{C}(#1)}
\def\R{\mathcal{R}}
\def\TL#1{\mathcal{T}(#1)}
\def\HL#1{\mathcal{H}(#1)}
\def\ET#1{\mathsf{ExecTime(#1)}}
\newcommand\mdoubleplus{\mathbin{+\mkern-10mu+}}

\AtBeginSection[]
{
    \begin{frame}
        \frametitle{Table of Contents}
        \tableofcontents[currentsection]
    \end{frame}
}

% context
\AtBeginSection[]
{
    \begin{frame}
        \frametitle{Table of Contents}
        \tableofcontents[currentsection]
    \end{frame}
}
\author[Renato Neves]{Renato Neves}

% logos of institutions
\titlegraphic{
  \begin{textblock*}{5cm}(6.7cm,7.57cm)
     \includegraphics[scale=0.05525]{./images/uminho.png}
  \end{textblock*}
  \begin{textblock*}{5cm}(9.4cm,7.57cm)
    \includegraphics[scale=0.50]{./images/haslab.pdf}
  \end{textblock*}
}

% No date
\date{}


\begin{document}

\title{Simply Typed Lambda-calculus}

\frame[plain]{\titlepage}

\section{The Calculus} 

\begin{frame}{Deductive Reasoning}

  \begin{block}{The essence}
          Knowledge obtained via \alert{\underline{assumptions}} and
          \alert{\underline{logical rules}}
  \end{block}

  \pause
  \bigskip
  Studied since Aristotle  \dots

  \dots \, long before the age of \alert{\underline{artificial}} computers

  What does it have to do with programming ?
\end{frame}

\begin{slide}{A Basic Deductive System}
  \begin{minipage}[0.3\textheight]{\textwidth}
  \begin{columns}[c]
  \begin{column}{0.5\textwidth}
          $\typeA,\typeB \dots$ denote \alert{\underline{propositions}}
          and $\typeOne$ a proposition that always holds   
  \end{column}
  \begin{column}{0.45\textwidth}
        \includegraphics[scale=0.2]{images/gentzen.jpg}
        \includegraphics[scale=0.16]{images/prawitz.jpg}
  \end{column}
  \end{columns}
  \end{minipage}

  If $\typeA$ and $\typeB$ are propositions then
  \begin{itemize}
    \item $\typeA \times \typeB$ is a proposition -- 
        conjunction of $\typeA$ and $\typeB$
    \item $\typeA \to \typeB$ is a proposition -- 
            implication of $\typeB$ from $\typeA$
  \end{itemize}
  
\end{slide}

\begin{frame}{A Basic Deductive System}
  
  $\Gamma$ denotes a \alert{\underline{list of propositions}} (often called context)

  $\Gamma \vljud \typeA$ reads \emph{``if the propositions in $\Gamma$ hold then  $\typeA$ also holds''}

  \small{
  \begin{flalign*}
      \infer[(\rulename{ass})]{\Gamma \vljud \typeA}{\typeA \in \Gamma} \hspace{0.8cm}
      \infer[(\rulename{trv})]{\Gamma \vljud  1}{}  \hspace{0.8cm}
      \infer[(\rulename{\pi_1})]{\Gamma \vljud \typeA}{\Gamma \vljud \typeA \times \typeB} \hspace{0.8cm}
      \infer[(\rulename{\pi_2})]{\Gamma \vljud \typeB}{\Gamma \vljud \typeA \times \typeB}
  \end{flalign*}
  \begin{flalign*}
      \infer[(\rulename{prd})]{\Gamma \vljud \typeA \times \typeB}
      {\Gamma \vljud \typeA \qquad \Gamma \vljud \typeB} \hspace{0.4cm}
      \infer[(\rulename{cry})]{\Gamma \vljud \typeA \to \typeB}{\Gamma, \typeA \vljud  \typeB}
      \hspace{0.4cm}
      \infer[(\rulename{app})]{\Gamma \vljud \typeB}
      {\Gamma \vljud  \typeA \to \typeB \quad
        \Gamma \vljud  \typeA}
  \end{flalign*}
        } 
  
  \bigskip
  \begin{block}{Exercise}
        Show that $\typeA \times \typeB \vdash \typeB \times \typeA$
  \end{block}
 
\end{frame}

\begin{slide}{New Knowledge From Old}
  The rules below are \alert{\underline{derivable}} from the previous system

  \begin{flalign*}
      \infer[\text{(exchange)}]{\Gamma, \typeB, \typeA,\Delta \vljud \typeC}{
      \Gamma, \typeA, \typeB,\Delta \vljud \typeC}
      \hspace{2cm}
      \infer[\text{(weakening)}]{\Gamma,\typeB \vljud \typeA}{\Gamma \vljud \typeA}
  \end{flalign*}
  \begin{flalign*}
      \infer[\text{(cut elimination)}]{\Gamma\vljud \typeB}{
      \Gamma, \typeA \vljud \typeB \qquad \Gamma \vljud \typeA}
  \end{flalign*}

  Proofs (again) by an appeal to your old friend \dots\  \alert{\underline{induction}} :-)
\end{slide}

\begin{frame}{Exercises}
  Derive the following judgements
  \begin{itemize}
          \item $\typeA \to \typeB,\typeB \to \typeC \vljud \typeA \to \typeC$ 
          \item $\typeA \to \typeB,\typeA \to \typeC \vljud \typeA \to \typeB
                  \times \typeC$
  \end{itemize}        
\end{frame}

\begin{frame}[plain]

        \begin{center}
                Back to programming \dots
        \end{center}
\end{frame}
\begin{slide}{The Bare Essentials of Programming}
  We should think of
  what are the \alert{\underline{basic features}} of programming \dots
  \begin{itemize}
  \item variables  
  \item function application and creation
  \item pairing \dots
  \end{itemize}
  and base our study on the \alert{\underline{simplest language}} with
  such features \dots
  \caixa{
    \begin{center}
        Simply-typed $\lambda$-calculus
    \end{center}
  }
  \vfill
  The basis of \alert{\texttt{\underline{Haskell}}}, \texttt{ML},
  \texttt{Eff}, \texttt{F\#}, \texttt{Agda}, \texttt{Elm} and many
  other programming languages
\end{slide}

\begin{frame}{Simply-typed $\lambda$-Calculus}

  Types are defined by $\typeA ::= \typeOne \mid \typeA \times \typeA \mid \typeA
  \to \typeA$

  $\Gamma$ now a \alert{\underline{non-repetitive}} list of typed variables
  ($x_1 : \typeA_1 \dots x_n : \typeA_n$)

  Programs built according to the following \alert{\underline{deduction rules}}

  \small{
  \begin{flalign*}
      \infer[(\rulename{ass})]{\Gamma \vljud x : \typeA}{x :
        \typeA \in \Gamma} \hspace{1.2cm}
      \infer[(\rulename{triv})]{\Gamma \vljud \ast : 1}{}  \hspace{1.2cm}
      \infer[(\rulename{\pi_1})]{\Gamma \vljud \pi_1 \, t : \typeA }
      {\Gamma \vljud t: \typeA \times \typeB}
  \end{flalign*}
  \begin{flalign*}
     \infer[(\rulename{prd})]{\Gamma \vljud \langle t,s \rangle :
        \typeA \times \typeB}
      {\Gamma \vljud t: \typeA \qquad \Gamma \vljud s: \typeB} \hspace{1cm}
      \infer[(\rulename{cry})]{\Gamma \vljud \lambda x : \typeA . \, t : \typeA
        \to \typeB}
      {\Gamma, x : \typeA \vljud t : \typeB}
    \end{flalign*}
   \begin{flalign*}
      \infer[(\rulename{app})]{\Gamma \vljud t \, s : \typeB}
      {\Gamma \vljud  t : \typeA \to \typeB \quad
        \Gamma \vljud  s : \typeA}
    \end{flalign*}
   }
\end{frame}

\begin{slide}{Examples of $\lambda$-terms}
  \begin{flalign*}
          & x : \typeA \vljud x : \typeA && \text{(identity)}
          \\[15pt]
          & x : \typeA \vljud \langle x, x \rangle : \typeA \times \typeA
          && \text{(duplication)} 
          \\[15pt]
          & x : \alert{\typeA \times \typeB}
          \vljud \langle \pi_2\ x , \pi_1\ x \rangle : \alert{ \typeB \times \typeA }
          && \text{(swap)}
          \\[15pt]
          & f : \alert{\typeA \to \typeB}, g : \alert{\typeB \to \typeC} \vljud
          \lambda x : \typeA . \> g (f \> x) : \alert{\typeA \to \typeC}
          && \text{(composition)}
  \end{flalign*}
\end{slide}

\begin{frame}{Exercises}

    Recall the derivations that lead to the judgement
    \[
      \typeA \to \typeB,  \typeA \to \typeC \vljud  \typeA 
      \to \typeB \times \typeC
    \]
    Build the corresponding program

    \bigskip
    Derive as well the judgement 
    \[
        \typeA \to \typeB \vljud  \typeA  \times \typeC
        \to \typeB \times \typeC
    \]
    and subsequently build the corresponding program
\end{frame}

\section{Denotational Semantics}

\begin{slide}{A Semantics for Simply Typed $\lambda$-calculus}
  We wish to assign a \alert{\underline{mathematical meaning}} to $\lambda$-terms
  \begin{flalign*}
    \sem{-} \, : \lambda\text{-terms} \longrightarrow ...
  \end{flalign*}
  so that we can reason about them rigorously, and take advantage
  of known mathematical theories
  \vfill \pause
  
  This is the goal of the next slides. But first \dots
\end{slide}

\begin{slide}{Functions: Basic Facts}

  For every set $X$ there exists a `trivial' function
  \begin{flalign*}
    ! : X \longrightarrow \{ \star \} = 1 \hspace{1cm} ! \, (x) = \star
  \end{flalign*}

  We can always pair two functions into
  $f : X \to A$, $g : X \to B$
  \begin{flalign*}
    \pv{f,g} : X \to A \times B \hspace{1cm} \pv{f,g}(x) = (f \> x, g \> x)
  \end{flalign*}

  There exist projection functions
  \begin{flalign*}
    \pi_1 : X \times Y \to X  \hspace{1cm} \pi_1(x,y) = x \\
    \pi_2 : X \times Y \to Y  \hspace{1cm} \pi_2(x,y) = y
  \end{flalign*}
\end{slide}

\begin{slide}{Functions: Basic Facts}

  We can always `curry' a function $f : X \times Y \to Z$ into
  \begin{flalign*}
    \lambda f : X \to Z^Y \hspace{1cm} \lambda f (x) = (y \mapsto f(x,y))
  \end{flalign*}

  Consider sets $X,Y, Z$. There exists an application function
  \begin{flalign*}
    \mathrm{app} : Z^Y  \times Y \to Z \hspace{1cm}
    \mathrm{app}(f,y) = f \> y
  \end{flalign*}
\end{slide}

\begin{slide}{Denotational Semantics}
  Types $\typeA$ interpreted as \alert{\underline{sets}} $\sem{\typeA}$
  \begin{flalign*}
    \sem{1} & = \{ \star \} \\
    \sem{\typeA \times \typeB} & = \sem{\typeA} \times \sem{\typeB} \\
   \sem{\typeA \to \typeB} & = \sem{\typeB}^{\sem{\typeA}} 
 \end{flalign*}

 Typing contexts $\Gamma$ interpreted as Cartesian products
 \begin{flalign*}
  \sem{\Gamma}\ = \sem{x_1 : \typeA_1, \dots, x_n : \typeA_n}\ =
 \sem{\typeA_1} \times \dots \times \sem{\typeA_n} 
 \end{flalign*}

 $\lambda$-terms $\Gamma \vljud t : \typeA$ interpreted as \alert{\underline{functions}}
 \begin{flalign*}
   \sem{\Gamma \vljud t : \typeA}\ : \sem{\Gamma}\ \longrightarrow \sem{\typeA}
 \end{flalign*}

\end{slide}

\begin{slide}{Denotational Semantics}
  $\lambda$-term $\Gamma \vljud t : \typeA$ interpreted as a function
 \begin{flalign*}
   \sem{\Gamma \vljud t : \typeA}\ : \sem{\Gamma}\ \longrightarrow \sem{\typeA}
 \end{flalign*}
 \vspace{-0.5cm}
 \small{
   \begin{flalign*}
      \infer[]{\sem{\Gamma \vljud x_i : \typeA} = \pi_i}{x_i :
        \typeA \in \Gamma} \hspace{1.2cm}
      \infer[]{\sem{\Gamma \vljud \ast : 1}\ =\ !}{}  \hspace{1.2cm}
      \infer[]{\sem{\Gamma \vljud \pi_1 t : \typeA} = \pi_1 \comp f }
      {\sem{\Gamma \vljud t : \typeA \times \typeB} = f}
  \end{flalign*}
  \begin{flalign*}
      \infer[]{\sem{\Gamma \vljud \langle t,s \rangle :
        \typeA \times \typeB}\ = \pv{f,g}}
    {\sem{\Gamma \vljud t: \typeA}\ = f \qquad \sem{\Gamma \vljud s: \typeB}\ = g}
    \hspace{0.3cm}
     \infer[]{\sem{\Gamma \vljud \lambda x : \typeA . \, t : \typeA
        \to \typeB}\ = \lambda f}
      {\sem{\Gamma, x : \typeA \vljud t : \typeB}\ = f}
    \end{flalign*}
   \begin{flalign*}
      \infer[]{\sem{\Gamma \vljud t \, s : \typeB}\ = \mathrm{app} \comp \pv{f,g}}
      {\sem{\Gamma \vljud  t : \typeA \to \typeB}\ = f \quad
        \sem{\Gamma \vljud  s : \typeA}\ = g}
    \end{flalign*}
}
\end{slide}

\begin{frame}{The Unravelling}
        \begin{align*}
                & \sem{x \vljud \pv{ \pi_2 \, x, \, \pi_1 \, x} } &  = \dots 
                &
                \\
                & \sem{- \vljud \lambda x. \, \pv{ \pi_2 \, x, \, \pi_1 \, x} } & = \dots 
                &
                \\[10pt]
                & \sem{f, g , x \vljud g \, f \, x} & = \dots
                &
                \\
                & \sem{f, g \vljud \lambda x. \, g \, f \, x} & = \dots
                &
                \\[10pt]
                & \sem{f, x \vljud \pv{f \, \pi_1 \, x, \, \pi_2 \, x} \, } & = \dots
                &
                \\
                & \sem{f \vljud \lambda x. \, \pv{f \, \pi_1 \, x, \, \pi_2 \, x} \, } & = \dots
                &
                \\
                & \sem{- \vljud \lambda f. \, \lambda x. \, \pv{f \, \pi_1 \,
                x, \, \pi_2 \, x} \, } & = \dots &
        \end{align*}
        \vfill
        \begin{center}
                (\scriptsize{\textbf{N.B.} all types omitted for simplicity})
        \end{center}
\end{frame}
\begin{slide}{Denotational Semantics and Program Equivalence Revisited}
  Show that the following equations hold
  \begin{flalign*}
    \sem{x, y \vljud \pi_1 \pv{x,y}}\ & =\
    \sem{x , y \vljud x } \\[5pt]
    \sem{\Gamma \vljud t}\ & =\
    \sem{\Gamma \vljud \langle \pi_1\ t, \pi_2\ t \rangle }
    \\[5pt]
    \sem{ x \vljud (\lambda y. \, \pv{x,y})\>  \> x  }\,
                                    & =\
                                    \sem{ x \vljud \pv{x,x}  }
  \end{flalign*}

  \pause
  Show that the (complicated) $\lambda$-term below is really just the identity
  \begin{align*}
    z \vljud \lambda x. \, \pv{\pi_2 \, x, \pi_1 \, x}\>
    \Big (\lambda y. \, \pv{\pi_2 \, y, \pi_1 \, y} \> z \Big ) 
  \end{align*}

  \pause
  \textbf{ Hard ? }
\end{slide}

\section{Equational System}

\section{Disjunctive Types}
\end{document}
