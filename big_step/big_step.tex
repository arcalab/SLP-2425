\documentclass{beamer}
\usepackage{etex} % fixes new-dimension error
\usepackage{lmodern}
\usepackage[T1]{fontenc}
\usetheme{metropolis}
\metroset{block=fill}
%\usetheme{Boadilla}

\setbeamertemplate{footline}
{
  \leavevmode%
  \hbox{%
  \begin{beamercolorbox}[wd=.4\paperwidth,ht=2.25ex,dp=1ex,center]{author in head/foot}%
    \usebeamerfont{author in head/foot}\insertshortauthor
  \end{beamercolorbox}%
  \begin{beamercolorbox}[wd=.5\paperwidth,ht=2.25ex,dp=1ex,center]{title in head/foot}%
    \usebeamerfont{title in head/foot}\insertsection
  \end{beamercolorbox}%
  \begin{beamercolorbox}[wd=.1\paperwidth,ht=2.25ex,dp=1ex,right]{date in head/foot}%
    \insertframenumber{} / \inserttotalframenumber\hspace*{2ex} 
  \end{beamercolorbox}}%
  \vskip0pt%
}
\makeatother
\setbeamertemplate{navigation symbols}{}
%----------------------------------------------------------------------------
\usepackage{graphicx,amsmath}
\usepackage{stmaryrd} % cf. interleave
\usepackage{booktabs}
\usepackage{amscd}
\usepackage{multicol}
\usepackage[absolute,overlay]{textpos}
\usepackage{alltt}
\usepackage{proof}
\usepackage{listings}
\usepackage[all]{xy}
\input{macros}
%------ using pstricks (rnode etc) ------------------------------------------
\usepackage{pstricks,pst-node,pst-text,pst-3d}

% ------ using color ---------------------------------------------------------
\newrgbcolor{goldenrod}{.80392 .60784 .11373}
\newrgbcolor{darkgoldenrod}{.5451 .39608 .03137}
\newrgbcolor{brown}{.15 .15 .15}
\newrgbcolor{darkolivegreen}{.33333 .41961 .18431}
\def\gold#1{{\goldenrod #1}}
\def\dgold#1{{\alert{#1}}}
\def\dkb#1{{\blue #1}}
\def\tdkb#1{\textbf{\darkblue #1}}
\def\gre#1{{\darkolivegreen #1}}
\def\gry#1{{\gray #1}}
\def\rdb#1{{\red #1}}
%----------------------------------------------------------------------------

\AtBeginSection[]
{
    \begin{frame}
        \frametitle{Table of Contents}
        \tableofcontents[currentsection]
    \end{frame}
}

\author[Renato Neves]{Renato Neves}

% logos of institutions
\titlegraphic{
  \begin{textblock*}{5cm}(6.5cm,7.4cm)
     \includegraphics[scale=0.06]{images/uminho.png}\hspace*{.85cm}~%
  \end{textblock*}
  \begin{textblock*}{5cm}(9.2cm,7.45cm)
    \includegraphics[scale=0.50]{images/haslab.pdf}
  \end{textblock*}
}

% No date
\date{}

\begin{document}

\title{Big-step Semantics}

\frame[plain]{\titlepage}

\section{Outline}

\begin{frame}{Semantics for every season}

        \hspace*{+5pt}\makebox[\linewidth][c]{%
        \begin{tabular}{ l l }
        \alert{\underline{Operational semantics}} & How a program operates
        \\
        Denotational semantics & What a program is 
        \\
        Axiomatic semantics & Which logical properties a program satisfies
        \end{tabular}
        }

\end{frame}

\begin{frame}{Motivation}

        The semantics $\longrightarrow^\star$ provides a notion
        of \alert{\underline{program equivalence}}

        \[
                \prog{p} \equiv \prog{q}
                \text{ iff }
                \Big (\pv{\prog{p}}{\sigma} \longrightarrow^\star v \text{ iff }
                \pv{\prog{q}}{\sigma} \longrightarrow^\star v \Big )
        \]


        \pause
        This leads us to a previous slide \dots
\end{frame}

\begin{frame}{ The search for meaning }
    \begin{examples}
    \begin{itemize}
            \item $\mathtt{p ; (q ; r)} \stackrel{?}{=} \mathtt{(p ; q) ; r} $
                    \\[10pt]
            \item $\mathtt{p \parallel q \stackrel{?}{=} q \parallel p} $ 
                    \\[10pt]
            \item $\mathtt{\left (p +_{\frac{1}{2}} q \right ) ; r 
                    \stackrel{?}{=} p ; r +_{\frac{1}{2}} q ; r}$ 
                    \\[10pt]
            \item $\mathtt{entangle(x,y) \stackrel{?}{=}}$ spooky action 
    \end{itemize}
  \end{examples}
\end{frame}

\begin{frame}{ Motivation }

        However (dis)proving equivalences via $\longrightarrow^\star$ is quite cumbersome

        Due to equivalence being concerned only with
        \alert{\underline{output}} \dots\ not intermediate steps \dots

        \dots\ and outputs obtained via $\longrightarrow^\star$
        relying on these

        \pause
        \bigskip
        Can we build a more direct, \alert{\underline{big-step}} semantics ?
\end{frame}

\section{Big-step semantics}

\begin{frame}{A simple while-language}

        \vspace{0.7cm}
	\begin{block}{Arithmetic expressions}
        $\prog{e} ::=  \mathtt{n}  \mid \mathtt{e \cdot e}
        \mid  \mathtt{x}  \mid \mathtt{e + e}$
	\end{block}

	\vspace{0.7cm}
	\begin{block}{Programs}
        $\prog{p} ::= \mathtt{x} := \mathtt{e} \mid
	\mathtt{p \> {\blue ;} \> p} \mid
	\mathtt{{\blue if} \> b \> {\blue then} \> p \> {\blue else} \> p} \mid
	\mathtt{{\blue while} \> b \> {\blue do} \> \{ \> p \> \}}$
	\end{block}

        \vfill
        \pause
        Homework: provide semantics to the arithmetic expressions
\end{frame}
\begin{frame}{A while-language and its semantics}
        \[
                \infer[(\text{asg})]{
                        \langle \mathtt{x := e}, \sigma \rangle \Downarrow 
                        \sigma[\mathtt{v} / \mathtt{x}]
                }{
                       \langle \mathtt{e}, \sigma \rangle \Downarrow \mathtt{v}
                } \hspace{1cm}
                \infer[(\text{seq})]{
                        \langle \mathtt{p \> {\blue ;} \> q}, \sigma \rangle \Downarrow \sigma''
                }{
                        \langle \mathtt{p}, \sigma \rangle \Downarrow \sigma' \qquad
                        \langle \mathtt{q}, \sigma' \rangle \Downarrow \sigma'' 
                }
        \]\vspace{0.001cm}
        \[\hspace{-0.6cm}
                \infer[(\text{if$_1$})]{
                        \langle \mathtt{{\blue if} \> b \> {\blue then} \> \> p \> {\blue else} \> q}, 
                        \sigma \rangle \Downarrow \sigma'
                }{
                        \langle \mathtt{b}, \sigma \rangle \Downarrow \mathtt{tt} \qquad
                        \langle \mathtt{p}, \sigma \rangle \Downarrow \sigma'
                } \hspace{1cm} 
                \infer[(\text{if$_2$})]{
                        \langle \mathtt{{\blue if} \> b \> {\blue then} \> \> p \> {\blue else} \> q}, 
                        \sigma \rangle \Downarrow \sigma'
                }{
                        \langle \mathtt{b}, \sigma \rangle \Downarrow \mathtt{ff} \qquad
                        \langle \mathtt{q}, \sigma \rangle \Downarrow \sigma'
                } 
        \]\vspace{0.001cm}
        \[
                \infer[(\text{wh$_1$})]{
                        \langle \mathtt{{\blue while} \> b \> {\blue do} \> \{ \> p \> \}}, \sigma \rangle
                        \Downarrow \sigma''
                }{
                        \langle \mathtt{b}, \sigma \rangle \Downarrow \mathtt{tt} \qquad
                        \langle \mathtt{p}, \sigma \rangle \Downarrow \sigma' \qquad
                        \langle \mathtt{{\blue while} \> b \> {\blue do} \> \{ \> p \> \}}, \sigma'
                        \rangle \Downarrow \sigma'' 
                }
        \]\vspace{0.001cm}
        \[
                \infer[(\text{wh$_2$})]{
                        \langle \mathtt{{\blue while} \> b \> {\blue do} \> \{ \> p \> \}}, \sigma \rangle
                        \Downarrow \sigma
                }{
                        \langle \mathtt{b}, \sigma \rangle \Downarrow \mathtt{ff}
                }
        \]
\end{frame}

\begin{frame}{The semantics at work}
        Program $\mathtt{x := x + 1 \> {\blue ;} \> x := x + 2}$ corresponds to the 
        \alert{\underline{syntax}} tree 
        \[
                \xymatrix@C=10pt@R=10pt{
                        & (\> ; \>) \ar[dl] \ar[dr]  & \\
                        \mathtt{x := x + 1} & & \mathtt{x := x + 2} 
                }
        \]

        \vspace{0.4cm}
        Memory $\sigma = x \mapsto 3$ yields the \alert{\underline{derivation}} tree
        \[
                \infer{
                        \langle \mathtt{ x := x + 1 \> {\blue ;} \> x := x + 2 }, \mathtt{x \mapsto 3}
                        \rangle \Downarrow \mathtt{x \mapsto 6}
                }{
                        \infer{
                                \langle \mathtt{x := x +1}, \mathtt{x \mapsto 3} \rangle 
                                \Downarrow \mathtt{x \mapsto 4}
                        }{
                                \langle \mathtt{x + 1}, \mathtt{x \mapsto 3} \rangle \Downarrow
                                \mathtt{4}
                        }\qquad \qquad                        
                        \infer{
                                \langle \mathtt{x := x + 2}, \mathtt{x \mapsto 4} \rangle 
                                \Downarrow \mathtt{x \mapsto 6}
                        }{
                                \langle \mathtt{x + 2}, \mathtt{x \mapsto 4} \rangle \Downarrow
                                \mathtt{6}
                        }
                }
        \]
\end{frame}

\begin{frame}{Exercise}
        Provide a big-step semantics to the propositional language

        \begin{block}{\vspace*{-3.5ex}}
        \[
                \prog{b} ::= \prog{x} \mid \prog{b} \wedge \prog{b} \mid \neg \prog{b}
        \]
        \end{block} 

\end{frame}

\begin{frame}{Equivalence of while-programs}
        The previous semantics yields the following notion of
        \alert{\underline{equivalence}} $\mathtt{p} \equiv \mathtt{q}$ if for
        all environments $\sigma$
        \begin{flalign*}
                \langle \mathtt{p}, \sigma \rangle \Downarrow \mathtt{\sigma'} 
                \text{ iff } \langle \mathtt{q}, \sigma \rangle \Downarrow \mathtt{\sigma'}
        \end{flalign*}

        Examples of equivalent terms
        \begin{itemize}
                \item $\mathtt{(p \> {\blue ;} \> q) \> {\blue ;} \> r} \equiv
                        \mathtt{p \> {\blue ;} \> (q \> {\blue ;} \> r)}$
                \item $\mathtt{({\blue if} \> b \> {\blue then} \> p \> {\blue else} \> q) 
                      \> {\blue ;} \> \prog{r}} \equiv 
                      \mathtt{{\blue if} \> b \> {\blue then} \> p \> {\blue ;} \> r \> 
                      {\blue else} \> q \> {\blue ;} \> r}$ 
        \end{itemize}
\end{frame}

\begin{frame}{The relation to the small-step semantics}

        \begin{lemma}
                \[
                        \pv{\prog{p}}{\sigma} \longrightarrow \pv{\prog{p}'}{\sigma'}
                        \Downarrow \sigma'' \text{ implies }
                        \pv{\prog{p}}{\sigma}
                        \Downarrow \sigma'' 
                \]
        \end{lemma}
        \begin{proof}
                \alert{\underline{Induction}} over the rules concerning $\longrightarrow$
        \end{proof}

        \pause
        \begin{theorem}
                \[
                        \pv{\prog{p}}{\sigma} \longrightarrow^\star \sigma'
                        \text{ iff }
                        \pv{\prog{p}}{\sigma} \Downarrow \sigma'
                \]
        \end{theorem}

        \begin{proof}
                Left-to-right-direction: previous lemma and \alert{\underline{induction}} over
                the rules concerning $\longrightarrow^\star$

                Right-to-left direction: \alert{\underline{induction}} over the
                rules concerning $\Downarrow$
        \end{proof}
\end{frame}

\bibliographystyle{amsalpha}
\bibliography{big_step}
\end{document}

